% Notwendige Start-Codezeile
\documentclass[a4paper, 12pt, twoside]{article}

% Präambel

% Pakete
\usepackage[left=3cm,right=2.5cm,bottom=3.5cm,top=2.5cm]{geometry} % Ok die Seitenrändereinstellungen passen so.
\usepackage{amsfonts}
\usepackage[utf8]{inputenc} % Kodierung
\usepackage[ngerman]{babel} % Sprache
\usepackage{amssymb}        % Mathematische Symbole wie z.b ganze Zahlen, reelle Zahlen etc.
\usepackage{amsmath}        % Um diverse mathematische Symbole nutzen zu können.
\usepackage{amsthm}         % Um Definitionen, Theoreme, Bemerkungen, Beispiele machen zu können.
\usepackage{mathtools}      % Dieses Paket liefert nützliche Werkzeuge, z.B "defined as equal" - sign uvm.
\usepackage{enumitem}       % Um individuelle Listen zu bauen.
\usepackage{xcolor}         % Um farbigen Text machen zu können. Diesen kann ich nutzen um wichtige persönliche Notizen hervorzuheben.
\usepackage{fancyhdr}       % Um saubere Kopf und Fußzeilen sowie Seitenzahlen zu erzeugen.
\usepackage{setspace}       % Damit kann ich Leerzeilen im Dokument einfügen.
\usepackage{graphicx}       % Einbinden von externen Bildern
% insbesondere beim Inhaltsverzeichnis nötig, da dieses sonst zu
% gequetscht aussieht.


% Symbolverzeichnis
%\usepackage{nomencl}
%\makenomenclature
%\renewcommand{\nomname}{Symbolverzeichnis}


% Globale Festlegungen
\setlength{\topsep}{4ex plus0.5ex minus0.5ex} % Festlegung: Größe der Absätze nach Definition, Theorem, etc.
\setlength\parindent{0pt} % Festlegung: Kein Einschub nach rechts.
\setstretch{1.2} % Festlegung: Zeilenabstand ist 1.2 statt 1.
\raggedbottom % twoside sorgt dafür, dass der Platz, der durch \\ und einer Leerzeile erzeugt wird sehr groß ist und nicht nur eine Leerzeile, was ich nämlich erzielen möchte.
              % Dieses Kommando sorgt dafür, dass dies wiederhergestellt wird und twoside trotzdem wirkt.

% Vorlagen
% 1) [before = \leavevmode\vspace{-\baselineskip}]
% // Um bei Theoremen mit Listen zu starten.

% Eigene Befehle

\newcommand\logeq{\mathrel{\vcentcolon\Longleftrightarrow}} % Dieser Befehl realisiert mittels "\logeq" ein "defnierendes Äquivalenzzeichen".
\newcommand{\ts}{\thinspace} % Damit ich nicht so viel schreiben muss, wenn ich kleine Leerzeilen hinzufügen möchte. Was ich oft tue bei Mengendefinitionen, da mir der Platz dort zu klein ist.

% Eigene Theoremstyles
% Format 1
\newtheoremstyle{Format1}
{\topsep}   % ABOVESPACE
{\topsep}   % BELOWSPACE
{\normalfont}  % BODYFONT
{0pt}       % INDENT (empty value is the same as 0pt)  % Das rückt den Kopf nach rechts ein
{\bfseries} % HEADFONT
{\newline}  % HEADPUNCT
{5pt plus 1pt minus 1pt} % HEADSPACE
{}          % CUSTOM-HEAD-SPEC


% Eigene Theorem-Umgebungen
\theoremstyle{Format1} % Alle Theoremumgebungen hierunter folgen den Spezifikationen vom "Format 1" - Theoremstil
\newtheorem{Def}{Definition}[section]       % Definition
\newtheorem*{Definition}{Definition}        % Definition (unnummeriert)
\newtheorem{Bsp}[Def]{Beispiel}             % Beispiel
\newtheorem{Bem}[Def]{Bemerkung}            % Bemerkung
\newtheorem{Satz}[Def]{Satz}                % Satz
\newtheorem*{Bez}{Bezeichnung}              % Bezeichnung (unnummeriert)
\newtheorem{Folg}[Def]{Folgerung}           % Folgerung
\newtheorem*{Folgerung}{Folgerung}          % Folgerung (unnummeriert)
\newtheorem{Lem}[Def]{Lemma}                % Lemma
\newtheorem*{Grundmodell}{Grundmodell}
\newtheorem*{Aussage}{Aussage}
\newtheorem*{Herleitung}{Herleitung}

% Code aus Stackexchange, welcher bei Nutzen von Norm oder Betrag automatisch passende Betragsgrößen generiert, also die Beträge dem Ausdruck entsprechend vergrößert:

\DeclarePairedDelimiter\abs{\lvert}{\rvert}%
\DeclarePairedDelimiter\norm{\lVert}{\rVert}%

% Swap the definition of \abs* and \norm*, so that \abs
% and \norm resizes the size of the brackets, and the
% starred version does not.
\makeatletter
\let\oldabs\abs
\def\abs{\@ifstar{\oldabs}{\oldabs*}}
%
\let\oldnorm\norm
\def\norm{\@ifstar{\oldnorm}{\oldnorm*}}
\makeatother

% Code aus Stackexchange: Sorgt dafür, dass die Abstände zwischen Zeilen in Allign Umgebungen etwas größer sind, also normal. Die Standardabstände sind mir zu gering.

\addtolength{\jot}{0.5em}


% Beginn des Dokumentes
\begin{document}

\newgeometry{} % Damit die Titelseite nicht von den Einstellungen des geometry packages beeinflusst wird.
% Titelblatt der Arbeit
\begin{titlepage}
	\begin{center}
		\vspace*{1cm}

		\Huge
		\textbf{Bachelorarbeit}

		\vspace{0.5cm}
		\LARGE
		Optimale Zuordnungen auf Geometrischen Graphen

		\vspace{1.5cm}
		\large Sebastian Koletzko\\

		\vspace{1cm}
		Datum: 17.12.23

		\vfill

		\vspace{5cm}

		%\includegraphics[width=0.4\textwidth]{RuhrU}

		\large
		Fakultät für Mathematik
		\\
		Ruhr-Universität Bochum
		\\
		\vspace{0.5cm}
		Prof. Dr. Maike Buchin
		\\
		Dr. Daniela Kacso

	\end{center}
\end{titlepage}

\restoregeometry % Damit die Titelseite nicht vom geoemtry-package beeinflusst wird - Endcodezeile.

\newpage\null\thispagestyle{empty}\newpage % Nach der Titelseite kommt immer eine leere Seite, denn die nachfolgende Seite, ist die Rückseite der Titelseite. Und ich möchte nicht auf der Rückseite direkt anfangen.

\thispagestyle{empty} % Damit keine Seitenzahl auf der Erklärung-Seite auftaucht.

\textbf{Eigenständigkeitserklärung:}
\\
Hiermit erkläre ich, dass ich die heute eingereichte Bachelorarbeit selbstständig verfasst und keine anderen als die angegebenen Quellen und Hilfsmittel benutzt sowie Zitate kenntlich gemacht habe.
Bei der vorliegenden Bachelorarbeit handelt es sich um in Wort und Bild völlig übereinstimmende Exemplare. Ich erkläre weiterhin, dass die vorliegende Arbeit noch nicht im Rahmen eines anderen Prüfungsverfahrens eingereicht wurde.
\\
\\
Witten, den 17.12.23
\\
\\
Sebastian Koletzko

\newpage
abstract:
In dieser Arbeit werden wir ein in [1] vorgestelltes Abstandsmaß auf geometrischen Graphen - den sogenannten \textit{min-sum graph distance} - untersuchen.
Dafür wird einleitend das notwendige mathematische Grundgerüst aus [2] formuliert und - darauf aufbauend - ein in Algorithmus
dargestellt, welcher den \textit{min-sum graph distance} zweier Bäume unter einer Einschränkung in der der Abbildung in Polynomialzeit berechnet.
Darüber hinaus zeigen wir die NP-Schwerheit der \textit{min-sum graph distance} für Graphen ohne die besagte Einschränkung in der Abbildung.


% Inhaltsverzeichnis. (Dieses wird nach Sektionen geordnet)
\newpage
\tableofcontents
\newpage\null\thispagestyle{empty}\newpage % Eine Leerseite nach dem Inhaltsverzeichnis.
\section{Einleitung}

Die geometrische Graphentheorie beschäftigt sich ...

Wir betrachten im Allgemeinen endliche, ungerichtete Graphen, welche in den euklidischen Raum eingebettet werden.

Geometrische Graphen in der euklidischen Ebene eignen sich in der Anwendung z.B. zur Approximation bzw. Rekonstruktion eines konkreten Netzwerkes.
Dabei interessieren wir uns für den Vergleich mehrerer solcher Darstellungen eines Netzwerks (wie z.B. eines Straßennetztes) unter dem Aspekt ihrer Güte bezüglich der Rekonstruktion.
\\
Eine zentrale Fragestellung dabei ist, wie man Abstandsmaße auf geometrischen Graphen definiert und unter welchen Bedingungen sich dieses Maß effizient berechnen lässt.
\\
Um die Ählichkeit zweier eingebetter Graphen zu beschreiben, haben Akitaya et al. \cite{Akitaya} neue Abstandsmaße eingeführt und ihre Eigenschaften untersucht.
\\
Besonderheit bei diesen Abstandmaßen ist, dass sie auf einer stetigen Zuordnung zwischen den zu vergleichenden Graphen basieren und somit auch die Topologie der Graphen berücksichtigen.
Das resultierende Maß beschreibt eine sogenannte "bottleneck-distance".
\\
Darauf aufbauend haben \cite{Buchin} Buchin et al. ein weiteres Abstandsmaß definiert, welches mögliche lokale Optimierungen miteinbezieht.
Vor dem Hintergrund, dass die "directed (weak) graph distance" eine bottle-neck distance beschreibt, wird bei der min-sum graph distance das Ziel verfolgt, lokale Abstände zu berücksichtigen.
[Ein kleiner Text über den Stand der Wissenschaft]

Mit diesem Abstandsmaß wollen wir uns in dieser Arbeit hauptsächlich beschäftigen.
In Kapitel 2 fassen wir dazu zunächst die Notation und die zentralen Begriffe aus [2] zusammen, welche den Grundstein für die Auseinandersetzung mit dem Min-Sum-Graph-Abstand bilden.
In Kapitel 3 werden wir den Min-Sum-Graph-Abstand motivieren und einführen.

Zur Berechnung des Min-Sum-Graph-Abstands haben Buchin et al haben in [1] einen Algorithmus beschrieben, welcher unter gewissen Einschränkungen in Polynomialzeit läuft.
Bevor wir uns abschließend diesem Algorithmus widmen, werden wir die NP-Schwerheit des Min-Sum-Graph-Abstands für allgemeine Graphen beweisen.
\newpage

\section{Grundlagen}

\begin{Def}
	Grundbegriffe
	Unter einem \textit{geometrischen Graphen} verstehen wir einen endlichen, ungerichteten Graphen $G=(V, E)$ zusammen mit seiner Einbettung als Teilmenge des $ \mathbb{R}^n $, wobei
    	\begin{enumerate}
		\item[1)] jeder Knoten $v \in V $ durch einen Punkt in $ \mathbb{R}^n $ repräsentiert wird und
		\item[2)] jede Kante ${u,v} \in E$ durch die Strecke der Punkte im $ \mathbb{R}^n $ repräsentiert wird.
    	\end{enumerate}

	Wir betrachten den $ \mathbb{R}^n $ hier grundsätzlich als metrischen Raum, versehen mit der euklidischen Norm.
	\\
	Wenn wir im weiteren Verlauf von einem geometrischen Graphen $G$ sprechen, so unterscheiden wir nicht explizit zwischen seiner abstrakten
	Beschreibung und seiner konkreten Einbettung als Teilmenge des euklidischen Raums.
	\\
	\\
	Für einen Knoten $v \in V$ definieren wir den \textit{$\varepsilon$-Ball} $B_{\varepsilon}(v)$ als die Menge
	$\{x \in \mathbb{R}^n.: \|v-x\| \leq \varepsilon\}$
	und für eine Kante $e \in E$ definieren wir den \textit{$\varepsilon$-Schlauch} $T_{\varepsilon}(e)$ als die Menge
	$\{x \in \mathbb{R}^n: \min_{a \in e}\|a-x\| \leq \varepsilon\}$.
\end{Def}

\begin{Def}
	Einfacher Weg in einem Geometrischen Graphen
	Ein \textit{einfacher Weg} in einem geometrischen Graphen $G$ ist eine stetige und injektive Abbildung $ \phi: [0, 1] \to G$.
\end{Def}

\subsection{Abstände zwischen geometrischen Graphen}

Seien $ G_1=(V_1, E_1) $ und $ G_2=(V_2, E_2) $ für den weiteren Verlauf geometrische Graphen mit
$n_1 = |V_1|, m_1 = |E_1|, n_2 = |V_2|$ und $m_2 = |E_2|$.
\begin{Def}
	Graph-Zuordnung (engl. graph mapping)
	Wir nennen eine Abbildung $s: G_1 \to G_2 $ eine \textit{Graph-Zuordnung}, falls
    	\begin{enumerate}
		\item[1)] s jeden Knoten $ v \in V_1 $ auf einen Punkt einer Kante von $ G_2 $ abbildet und
		\item[2)] s alle Kanten $ {u,v} \in E_1 $ auf einen einfachen Weg in $G_2$ mit dem Startpunkt $s(u)$ und dem Endpunkt $s(v)$ abbildet.
    	\end{enumerate}

	Eine Graph-Zuordnung s definiert somit stetige Abbildung von $ G_1 $ nach $ G_2 $.\\
	Bemerkung: Eine Graph-Zuordnung ist im Allgemeinen weder injektiv noch surjektiv.
	$ G_1 $ und $ G_2 $ müssen daher insbesondere nicht homeomorph sein.

\end{Def}

TODO: Hier wäre ein Beispiel angebracht

\begin{Def}
	Fréchet-Abstand (engl. fréchet distance)
	Für zwei Kurven $ f, g: [0,1] \to \mathbb{R}^n $ definieren wir den \textit{Fréchet-Abstand} von $f$ und $g$ als
	$$ \delta_F(f,g) =  \inf_{\sigma:[0,1] \to [0,1]} \; \max_{t \in [0,1]} \lVert f(t)-(g(\sigma(t)) \rVert, $$
	wobei sich $\sigma $ über alle orientierungserhaltenen Homeomorphismen erstreckt.
	\\
	\\
	Wir definieren den \textit{schwachen Fréchet-Abstand} von $f$ und $g$ als
	$$\delta_{wF}(f,g) =\inf_{\alpha , \beta :[0,1] \to [0,1]} \; \max_{t \in [0,1]} \lVert f(\alpha(t))-(g(\sigma(t)) \rVert,$$
	wobei sich $\alpha$ und $\beta$ über alle stetigen Abbildungen erstrecken, welche die Endpunkte fixieren, also $\alpha(0) = \beta(0) = 0$
	und $\alpha(1) = \beta(1) = 1$.
	\\
	\\
	Die Notation $ \delta_{(w)F} $ benutzen wir zugunsten der Kompaktheit nachfolgend, wenn wir im Kontext den Fréchet-Abstand sowie den schwachen Fréchet-Abstand zugleich -
	wir schreiben in diesem Fall auch \textit{(schwacher) Fréchet-Abstand} - adressieren wollen.
	\\
	\\
	Zur Anschauung des Fréchet-Abstands stelle man sich $f$ und $g$ als Kurven im $\mathbb{R}^2$ vor. Der (euklidische) Abstand zwischen $f$ und $g$ zum Zeitpunkt $t_0 \in [0,1]$ entspricht gerade der Länge der Strecke
	mit den Endpunkten $f(t_0)$ und $g(t_0)$. Bei dem Fréchet-Abstand dürfen wir eine Kurve reparametrisieren, um den maximal angenommenen euklidischen Abstand zwischen $f$ und $g$ zu minimieren.
	Wir werden so einen maximal angenommen Abstand sinngemäß auch als \textit{Flaschenhals-Abstand} bezeichnen.\\

	In der Literator TODO: REFERENZ findet man hierzu oft das Beispiel eines Mannes, der seinen Hund an einer Leine spazieren führt. (Siehe Skizze)
	Beide starten jeweils am Punkt $f(0)$ bzw. $g(0)$ und enden jeweils am Punkt $f(1)$ bzw. $g(1)$.

	Für den Fréchet-Abstand darf einer von beiden auf seinem festgelegten Weg - welcher der Spur der Kurve entspricht - zu jedem Zeitpunkt beliebig beschleunigen oder sogar stehenbleiben, aber nicht die Richtung wechseln.
	Dies mit dem Ziel, die dabei maximal angenommene Länge der Leine möglichst zu minimieren.
	Für den schwachen Fréchet-Abstand dürfen der Mann sowie sein Hund auf ihrem Weg beliebig beschleunigen und auch die Richtung ändern.
	TODO: Beispiel Illustration anfügen
	\\
	Da der Fréchet-Abstand symmetrisch ist !TODO: REFERENZ oder beweisen!, spielt es hierbei keine Rolle, welche der beiden Kurven wir reparametrisieren.
\end{Def}


\begin{Def}
	Abstände auf Graphen
	Wir definieren den \textit{gerichteten Graph-Abstand} $ \vec{\delta}_G $ als
	$$ \vec{\delta}_G(G_1,G_2) = \inf_{s: G_1 \to G_2} \: \max_{e \in E_1} \delta_F(e, s(e)) $$
	und den \textit{gerichteten schwachen Graph-Abstand} $ \vec{\delta}_{wG} $ als
	$$  \vec{\delta}_{wG}(G_1,G_2) = \inf_{s: G_1 \to G_2} \: \max_{e \in E_1} \vec{\delta}_{wF}(e, s(e)), $$
	wobei s sich über alle Graph-Zuordnungen erstreckt.
\end{Def}

\begin{Def}
	$\varepsilon$-Platzierung (engl. $\varepsilon$-Placement)
	Eine \textit{$\varepsilon$-Platzierung eines Knotens $v \in V_1$} ist eine maximal zusammenhängende Komponente von $G_2$ eingeschränkt auf $B_{\varepsilon}(v)$.
	Eine \textit{$\varepsilon$-Platzierung einer Kante $e = \{u,v\} \in E_1$} ist ein Weg $P$ in $G_2$, welcher eine Platzierung $C_u$ von $u$ mit einer Platzierung $C_v$ von $v$
	so verbindet, dass $\delta_F(e, P) \leq \varepsilon$.
	\\
	In diesem Fall nennen wir $C_u$ und $C_v$ \textit{voneinander erreichbar}.
	\\
	Eine \textit{$\varepsilon$-Platzierung von $G_1$} ist eine Graph-Zuordnung $s: G_1 \to G_2$, so dass $s$ jede Kante $e \in G_1$ auf eine $\varepsilon$-Platzierung von $e$ abbildet.
	\\
	Eine schwache $\varepsilon$-Platzierung einer Kante $e = \{u,v\} \in E_1$ ist ein Weg $P$ in $G_2$, welcher eine Platzierung $C_u$ von $u$ mit einer Platzierung $C_v$ von $v$
	so verbindet, dass $\delta_{wF}(e, P) \leq \varepsilon$.
	\\
	Eine \textit{$\varepsilon$-Platzierung von $G_1$} ist eine Graph-Zuordnung $s: G_1 \to G_2$, so dass $s$ jede Kante $e \in G_1$ auf eine $\varepsilon$-Platzierung von $e$ abbildet.
	In diesem Fall nennen wir $C_u$ und $C_v$ \textit{schwach voneinander erreichbar}.
	\\
	Eine schwache \textit{$\varepsilon$-Platzierung von $G_1$} ist eine Graph-Zuordnung $s: G_1 \to G_2$, so dass $s$ jede Kante $e \in G_1$ auf eine schwache $\varepsilon$-Platzierung von $e$ abbildet.
\end{Def}

\subsection{Lokale Optimierungen von Graph-Zuordnung}

Grundsätzlich können viele Graph-Zuordnungen $s: G_1 \to G_2$ existieren, welche den gerichteten (schwachen) Graph-Abstand einhalten.
Aufgrund der inherenten Eigenschaft des gerichteten (schwachen) Graph-Abstands, ebenfalls einen Flaschenhals-Abstand zu beschreiben, gibt uns dieses Maß im
Allgemeinen relativ wenig Aufschluss über die Güte einzelner Zuordnungen der Kanten in $G_1$ unter der Graph-Zuordnung $s$.
\\
\\
Im Anwendungsfall kann man sich z.B. auch mehrere Rekonstruktionen $G_{1,1}, ..., G_{1,n}$ eines Netzwerkes $G_2$ vorstellen.
Naheliegend wäre dabei die Frage, welche der Rekonstruktionen dem urpsprünglichen Netzwerk $G_2$ am ähnlichsten ist.
Der gerichtete (schwache) Graph-Abstand zwischen einem $G_{1i}$ $(i \in \{1,...,n\})$ und $G_2$ gibt uns dabei relativ wenig Aufschluss.
Im schlimmsten Fall haben evlt. sogar alle $G_{1i}$ denselben Flaschenhals-Abstand zu $G_2$, da sie denselben (entscheidend) großen Fehler bei der Rekonstruktion von $G_2$ haben.
\\
\\
Dies motiviert die Einführung eines weiteren Abstandsmaßes auf Graphen, welches auch lokale Optimierungen der Zuordnungen von Kanten in Betracht zieht.
\\
\\
Wir merken jedoch an dieser Stelle an, dass im Allgemeinen eine Graph-Zuordnung $\tilde{s}: G_1 \to G_2$, die jede Kante $e \in E_1$ optimal bezüglich des (schwachen)
Fréchet-Abstands auf $G_2$ abbildet, so dass $\delta_{(w)F}(e, \tilde{s}(e)) \leq \delta_{(w)F}(e, s(e))$, nicht existiert (siehe dazu [2], Seite xx).
\\
\\
Als Alternative bietet es sich an, die Summe der Abstände zwischen den Kanten in $G_1$ und ihren Bildern in $G_2$ unter der Graph-Zuordnung als Optimalitätskriterum zu betrachten.

\section{Der Min-Sum-Graph-Abstand}

\begin{Def}
	Min-Sum-Graph-Abstand (engl. min-sum graph distance)

	Für die Graphen $G_1$ und $G_2$ definieren wir ihren \textit{Min-Sum-Graph-Abstand}$_{dist}(G_1, G_2)$ als
	$$\min_{s: G_1 \to G_2} \sum_{e \in E_1} dist(e, s(e))$$
	wobei sich $s$ über alle Graph-Zuordnungen von $G_1$ nach $G_2$ erstreckt und $dist$ ein für den Vergleich von $e \in E_1$ mit $s(e)$
	passendes Abstandsmaß (z.B. zwischen Kurven im euklidischen Raum) repräsentiert.
	\\
	\\
	Verwenden wir den (schwachen) Fréchet-Abstand als Abstandsmaß zwischen Kurven, so bezeichnen wir entsprechend den \textit{Min-Sum-Graph-Abstand}$_{(w)F}(G_1, G_2)$ als
	$$\min_{s: G_1 \to G_2} \sum_{e \in E_1} \delta_{(w)F}(e, s(e)).$$
	\\
	\\
	Prinzipiell fordern wir nicht, dass $s$ dabei den (schwachen) gerichteten Graph-Abstand zwischen $G_1$ und $G_2$ einhält.
	Es macht im Einzelfall jedoch durchaus Sinn, sich bei der Bestimmung des \textit{Min-Sum-Graph-Abstands} lediglich auf die Graph-Zuordnung zu beschränken,
	welche den gerichteten (schwachen) Graph-Abstand einhalten.
\end{Def}

TODO: Etwas zur Symmetrie schreiben.
\\
\\
\textit{Min-Sum-Graph-Abstand}$_{(w)F}(G_1, G_2)$ ist im Allgemeinen nicht symmetrisch.
\\
\\
\subsection{Zur Komplexität des Min-Sum-Graph-Abstands$_{(w)F}$}

Bevor wir uns mit dem erwähnten Polynomialzeit-Algorithmus beschäftigen, wollen wir zeigen, dass die Berechnung dese für allgemeine Graphen zeigen.
Ein ähnliches Setting finden wir in [2] (Seite). Hier wird die NP-Schwerheit des Graph-Abstands für allgemeine Graphen über eine Reduktion von
\textit{Binary Constraint Satifaction Problem (CSP)} gezeigt. In unserem Beweis für NP-Schwerheit orientieren wir uns im Wesentlichen an dieser Konstruktion.

\begin{Def}
	Binary Constraint Satisfaction Problem
	Ein \textit{Binary Constraint Satisfaction Problem} beschreibt folgendes Entscheidungsproblem:
	\\
	Gegeben eine Instanz $\langle X,D,C \rangle$ bestehend aus
	$$ \text{einer Menge von \textit{Variablen }}X = \{x_1, x_2, ..., x_n\},$$
	$$ \text{einer Menge von \textit{Domänen }}D = \{D_1, D_2, ..., D_n\} $$
	$$ \text{und einer Menge von \textit{Bedingungen }}C = \{C_1, C_2, ..., C_k\}. $$
	Für jede Variable $ x_i \in X$ beschreibt die Domäne $ D_i \in D$ die Menge ihrer möglichen Wertzuweisungen.
	Eine Bedingung $C_{i,j} \in C$ spezifiert für je zwei unterschiedliche Variablen $x_i, x_j \in X$ eine Relation $R_{C_{i,j}}$ $\subseteq D_v \times D_w$.
	Wir nennen ein Wertepaar $(d_i, d_j) \in D_i \times D_j$ für eine vorhandene Bedingung $C_{i,j}$ an die Variablen $x_i,x_j$ \textit{zulässig}, wenn $(d_i,d_j) \in R_{C_{i,j}}$,
	ansonsten \textit{verletzt} das Wertepaar $(d_i, d_j)$ die Bedingung $C_{i,j}$ und wir nennen $(d_i,d_j)$ \textit{unzulässig}.
	\\
	Die Fragestellung ist nun, ob alle Variablen einem Wert aus ihrer Domäne zugewiesen werden können, so dass keine Bedingung verletzt wird?
	\\
	\\
	TODO: Referenz, dass CSP NP-Schwer ist
\end{Def}

\begin{Satz}
	Theorem: Sei $\varepsilon \geq 0$. Das Entscheidungsproblem $$ \textit{Min-Sum-Graph-Abstand}_{(w)F}(G_1, G_2) \leq  \varepsilon $$ ist NP-Schwer.
\end{Satz}

Sei $\langle X,D,C \rangle$ ein beliebiges Binary Constraint Satisfaction Problem.
\\
\\
Wir repräsentieren jede Variable $x_i \in X$ durch einen Knoten $v_i \in V_1$ in $G_1$ und für jede Bedingung $C_{i,j}$ an die Variablen $x_i, x_j$
hat $G_1$ die Kante $\{v_i, v_j\}$. Wir setzen $\varepsilon = |E_1|$.
\\
Sei $B_1(v_k)$ der Ball mit dem Radius $r=1$ um einen Knoten $v_k$.
Wir betten $G_1$ so in die euklidische Ebene ein, dass sich für je zwei Knoten $x_i,x_j$ ihre $\varepsilon$ Bälle nicht berühren und
der $\varepsilon$-Schlauch zwischen $x_i$ und $x_j$ keinen $\varepsilon$-Ball eines anderen Knoten überlappt.
\\
\\
Dies lässt sich z.B. realisieren, indem wir die Knoten in $G_1$ (gleichmäßig) auf einem Kreis mit entsprechend großem Radius verteilen.

Jeden Wert $d_{i,a} \in D_i$ repräsentieren wir durch einen Knoten $u_{i,a}$ in $G_2$, welchen wir innerhalb des Balles $B_1(v_i)$ einbetten.
Für jedes Wertepaar $d_{i,a} \in D_i, d_{j,b} \in D_j$ enthält $G_2$ genau dann die Kante $\{u_{i,a},u_{j,b}\}$, wenn die Wertekombination
durch eine Bedingung an die Variablen $x_i$ und $x_j$ erlaubt ist.
\\
Alle einzubettenen Kanten ergeben sich direkt (als Strecken) zwischen den jeweiligen Knoten-Einbettungen und
damit behaupten wir, dass sich die oben beschriebene Einbettung von $G_1$ und $G_2$ in Polynomialzeit berechnen lässt.
\\
\\
TODO: Figure XX zeigt die Konstruktion anhand eines kleinen Beispiels
\\
\\
Wir wollen nun zeigen, dass
$$ \min_{s: G_1 \to G_2} \sum_{e \in E_1} \delta_{(w)F}(e, s(e)) \leq |E_1| * \tilde{\varepsilon} \iff \langle X,D,C \rangle \text{ ist lösbar}$$
"$\Leftarrow$":
Sei $\langle X,D,C \rangle$ lösbar.
\\
Dann existiert eine Wertezuweisung $(\tilde{d_1},\tilde{d_2},...,\tilde{d_n}) \in {D_1 \times D_2 \times ... \times D_n},$ welche keine der Bedingungen verletzt.
\\\\
Wir definieren eine Abbildung $\tilde{s}:G_1 \to G_2$ mit $\tilde{s}(x_i) = \tilde{d_i}$.
\\
\\
Da jeder Knoten $x_i \in V_1$ durch $\tilde{s}$ auf einen Knoten in $G_2$ abgebildet wird, liegt $\tilde{s}(x_i)$ in $G_2$.
\\
Zusätzlich impliziert eine Kante $\{v_i, v_j\} \in E_1$ eine Bedingung $C_{i,j}$ an die Variablen $x_i$ und $x_j$ in $\langle X,D,C \rangle$
Da $(\tilde{d_1},\tilde{d_2},...,\tilde{d_n})$ keine Bedingung verletzt, ist $(d_i,d_j) \in R_{C_{i,j}}$ und damit existiert per Konstruktion
die Kante $\{u_{\tilde{d_i}}, u_{\tilde{d_j}}\} \in E_2$.
Insgesamt ist damit $\tilde{s}$ eine Graph-Zuordnung.
\\
\\
Da $\tilde{s}$ Kanten in $G_1$ auf Kanten in $G_2$ abbildet, ist für jede Kante $\{x_i, x_j\}$ in $G_1$ ihr Bild $\tilde{s}(\{x_i,x_j\})$ per Konstruktion
von $G_1$ und $G_2$ eine $\tilde{\varepsilon}$-Kanten-Platzierung, damit gilt
$$\delta_{(w)F}(\{x_i,x_j\}, \tilde{s}({x_i,x_j})) \leq \tilde{\varepsilon}.$$
\\
Daraus folgt weiter, dass $$\sum_{e \in E_1} \delta_{(w)F}(e, s(e)) \leq |E_1| * \tilde{\varepsilon}. (i) $$
\\
Über die Identität $$\min_{s: G_1 \to G_2} \sum_{e \in E_1} \delta_{(w)F}(e, s(e)) \leq \sum_{e \in E_1} \delta_{(w)F}(e,\tilde{s}(e)) $$
und (i) erhalten wir
$$\min_{s: G_1 \to G_2} \sum_{e \in E_1} \delta_{(w)F}(e, s(e)) \leq |E_1| * \tilde{\varepsilon}$$
\\
"$\Rightarrow$":
\\
Sei $$\min_{s: G_1 \to G_2} \sum_{e \in E_1} \delta_{(w)F}(e, s(e)) \leq {\varepsilon}.$$
\\
\\
So ist $s(e)$ für alle $e \in E_1$ eine (schwache) $\varepsilon$-Platzierung von $e$,
da ansonsten $\delta_{(w)F}(\tilde{e}, s(\tilde{e})) > \varepsilon$ für ein $\tilde{e} \in E_1$ und damit insbesondere $\sum_{{e}\in E_1} \delta_{(w)F}(e, s(e)) > \varepsilon$.
(*)
\\
\\
Zusätzlich wollen wir argumentieren, dass $s(e)$ für alle $e \in E_1$ eine (schwache) 1-Platzierung von $e$ ist.
\\
Angenommen, für eine Kante $\{v_i,v_j\} \in E_1$ sei $s(\{v_i, v_j\})$ keine (schwache) 1-Platzierung von $\{v_i,v_j\}$.
So ist per Definition $\delta_{(w)F}(\{v_i,v_j\}, s(\{v_i, v_j\})) > 1$
und es trifft genau einer von 3 Fällen zu:

\begin{enumerate}
	\item[1)] $s(v_i) \notin B_1(v_i)$ und $s(v_j) \in B_1(v_j)$
	\item[2)] $s(v_i) \in B_1(v_i)$ und $s(v_j) \notin B_1(v_j)$
	\item[3)] $s(v_i) \notin B_1(v_i)$ und $s(v_j) \notin B_1(v_j)$
\end{enumerate}

Wir betrachten zunächst Fall 1:
\\
Wegen (*) liegt $s(v_i)$ innerhalb von  einer $\varepsilon$-Platzierung von $v_i$, die wir mit $C_i$ bezeichnen.
Per Konstruktion hat $C_i$ eine Teilkomponente, welche in $B_1(v_i)$ liegt und gerade der zugehörigen $1$-Platzierung von $v_i$ entspricht und die wir mit $\tilde{C_i}$ bezeichnen.
\\
\\
Wir defninieren $\tilde{s}: G_1 \rightarrow G_2$ wie folgt:
\begin{enumerate}
	\item[(i)] $\tilde{s}$ bildet $v_i$ auf einen beliebigen Punkt innherhalb von $\tilde{C_1}$ ab.
	\item[(ii)] Für alle anderen Knoten $v \in V_1$ ist $\tilde{s}(v)=s(v)$.
\end{enumerate}

$\tilde{s}$ ist eine Graph-Zuordnung, da $s$ nach Vorraussetzung eine Graph-Zuordnung ist und es sich bei $\tilde{s}(v_i)$ lediglich um eine Verschiebung
des Bildpunktes von $s(v_i)$ entlang einer zusammenhängenden Komponente von $G_2$ handelt.
\\
Ferner gilt:
\begin{enumerate}
	\item[1] $\delta_{(w)F}(e, \tilde{s}(e)) \leq \delta_{(w)F}(e, s(e))$, falls $v_i$ ein Knoten von $e$ ist, insbesondere ist
	\item[2] $\delta_{(w)F}(e, \tilde{s}(e)) < \delta_{(w)F}(e, s(e))$, falls $e = \{v_i, v_j\}$ und in allen anderen Fällen gilt
	\item[3] $\delta_{(w)F}(e, \tilde{s}(e)) = \delta_{(w)F}(e, s(e))$.
\end{enumerate}
Insgesamt folgt, dass
$$\sum_{{e}\in E_1} \delta_{(w)F}(e, \tilde{s}(e)) < \sum_{{e}\in E_1} \delta_{(w)F}(e, s(e)),$$
was im Widerspruch zu der Annahme steht, dass $s$ eine $\text{Min-Sum-Graph-Zuordnung}_{(w)F}$ ist.
\\
Die Argumentation für Fall 2 verläuft analog wie für Fall 1.
Für Fall 3 kombinieren wir die Argumentationen von Fall 1 und Fall 2, in welcher die scharfe Ungleichung im Allgemeinen allerdings erst gilt,
nachdem beide Bildpunkte $s(v_i)$ und $s(v_j)$ entsprechend versetzt wurden.
\\
\\
Damit ist für eine Kante $\{v_i, v_j\} \in E_1$ ihr Bild $s(\{(v_i, v_j)\})$ stets eine 1-Platzierung von $\{v_i, v_j\}$. (**)
\\
\\
Im Falle des Fréchet-Abstands gilt damit, dass wir durch eine $\text{Min-Sum-Graph-Zuordnung}_F$ $s: G_1 \to G_2$ das Bild Kante unter s
$s(\{v_i, v_j\})$ eindeutig mit einer Kante $\{u_{i,a}, u_{j,b}\} \in E_2$ identifizieren können. Dabei handelt es sich nämlich genau um die Kante, welche
die entsprechenden 1-Platzierungen der Knoten $v_i$ und $v_j$ verbindet und in $T_1(\{v_i, v_j\})$ liegt.
Durch (**) ist aufgrund der Konstruktion stets gewährleistet, dass für die Kante $\{u_{i,a}, u_{j,b}\} \in E_2$, ihren assoziierten Wertezuweisungen $(d_{i,a}, d_{j,b}) \in D_i \times D_j$
und die Bedingung $C_{i,j} \in C$ an die Variablen $x_i, x_j$ gilt:
$$(d_{i,a},d_{j,b}) \in R_{C_{i,j}}$$
Damit ist (im Falle des Fréchet-Abstands) $\langle X,D,C \rangle$ erfüllbar.
\\
\\
Für den schwachen Fréchet-Abstand gilt das Argument, dass wir das Bild einer Kante in $E_1$ unter $s$ eindeutig mit einer Kante in $E_2$ identifizieren können, allgemein nicht.

Dies liegt daran, dass grundsätzlich jedes beliebige Knotenpaar $u_{i,a}, u_{i,b} \in V_2$ mit $u_{i,a} \in B_1(v_i)$, $u_{i,b} \in B_1(v_j)$ eine Kante in $E_2$ haben kann.
\\
So existieren potentielle Kantenzüge bestehend aus mindestens 3 Kanten innerhalb von $T_1(\{v_i, v_j\}$, deren Startpunkt in $B_1(v_i)$ und deren Endpunkt in $B_1(v_j)$ liegt.
Ein solcher Kantenzug $P$ ist dabei stets auch eine schwache 1-Platzierung der Kante $\{v_i, v_j\}$, dass heißt
$$ \delta_{wF}(\{v_i, v_j\}, P) \leq 1.$$
\\
\\
TODO: Skizze
\\
\\
Um dies zu umgehen, fügen wir mittig auf jeder Kante $\{v_i, v_j\} \in E_1$ einen zusätzlichen Knoten $v_{ij}$ in $G_1$ ein,
so dass $B_1(v_{ij}) \cap B_1(v_i) = B_1(v_{ij}) \cap B_1(v_j) = \emptyset$.
\\
\\
Hinweis: Hierfür können wir initial auch $\varepsilon = \max \{m_1, 4\}$ fordern.
\\
\\
Die Kante $\{v_i, v_j\}$ zerfällt dabei in die zwei Kanten $\{v_i, v_{ij}\}$, $\{v_{ij},v_j\}$ und
jede 1-Platzierung von $\{v_i, v_j\}$ zerfällt dabei zwei 1-Platzierungen der Kanten $\{v_i, v_{ij}\}$ und $\{v_{ij}, v_j$\}.
\\
\\
TODO: Skizze
\\
\\
TODO: Ergänze Erklärung, warum das Problem jetzt gelöst ist.
\\
\\
Wir bezeichnen mit $\tilde{G_1}=(\tilde{V_1}, \tilde{E_1})$ den Graphen, den wir durch die oben beschriebene Transformation erhalten.
\\
Gilt nun
$$ \min_{s: \tilde{G_1} \to G_2} \sum_{e \in E_1} \delta_{wF}(e, s(e)) \leq 2m_1, $$
so erhalten wir für jede Kante $\{v_i, v_j\} \in E_1$ und ihre Zerlegung $\{v_i, v_{ij}\},\{v_{ij}, v_j\} \in \tilde{E_1}$
über $s(\{v_i, v_{ij}\},\{v_{ij}, v_j\})$ die eindeutige, assoziierte Kante $\{u_{i,a}, u_{i,b}\} \in E_2$.
Die durch $\{u_{i,a}, u_{j,b}\} \in E_2$ repräsentierte Wertezuweisung $(d_{i,a}, d_{j,b})$ verletzt dabei nicht die Bedingung an die
durch $v_i$ und $v_j$ repräsentierten Variablen $x_i, x_j$.
Daraus ergibt sich insgesamt eine Wertezuweisung, welche keine Bedingung verletzt.
\\
\\
Damit ist $\langle X,D,C \rangle$ lösbar. \qed
\\
\\
TODO: Eine Bemerkung zur der Verallgemeinerung des Beweises?
\\
\\
\section{Ein polynomieller Algorithmus}

Im Nachfolgenden werden wir zur Berechnung des Min-Sum-Graph-Abstands$_{(w)F}$ den Algorithmus aus [1] beschreiben und diesen anhand eines konkreten Beispiels ausführen.
Sei $G_1$ von nun an ein Baum, also kreisfrei und zusammenhängend.

\subsection{Einschränkungen an die Graph-Zuordnungen} \label{Einschränkungen}
In der Definition des Min-Sum-Graph-Abstands$_{(w)F}(G_1, G_2)$ ziehen wir alle Graph-Zuordnungen $s: G_1 \to G_2$ in Betracht.
Im weiteren Verlauf werden wir diese Menge mit zwei Einschränkungen versehen.
Die erste Einschränkung ist dabei optional und die zweite Einschränkung ist notwendig, um die polynomielle Laufzeit des Algorithmus zu gewährleisten.

\subsubsection{Erste Einschränkung}
Wir haben den Begriff des Min-Sum-Graph-Abstands primär vor dem Hintergrund motiviert, optimale Graph-Zuordnungen auf geometrischen Graphen zu betrachten,
die bereits einen Flaschenhals-Abstand wie den gerichteten (schwachen) Graph-Abstand einhalten.
Um diesen Bezug zu erhalten, werden wir die in Betrachtung zu ziehenden Graph-Zuordnungen zunächst auf die Menge
$S = \{s: G_1 \to G_2 \text{ $|$ } s$ ist eine (schwache) $\varepsilon$-Platzierung von $G_1$ nach $G_2\}$ beschränken.
\\
Grundsätzlich ist dabei $\varepsilon \geq \vec{\delta}_{(w)F}(G_1, G_2)$ zu fordern, da eine $\varepsilon$-Platzierung von $G_1$ nach $G_2$ für $\varepsilon < \vec{\delta}_{(w)F}(G_1,G_2)$
nicht existiert. Dies folgt direkt aus der Defintion des gerichteten (schwachen) Graph-Abstands.
\\
\subsubsection{Zweite Einschränkung} \label {Zweite Einschränkung}
Da wir hier zur Berechnung der Abstände zwischen einer Kante $\{u,v\} \in E_1$ und ihrem Bild $s(\{u,v\})$ unter einer Graph-Zuordnung $s$ den (schwachen) Fréchet-Abstand verwenden,
hat die Wahl der Bilder $s(u)$ und $s(v)$, innerhalb der entsprechenden $\varepsilon$-Platzierungen von $u$ und $v$, eine direkte Auswirkung auf die summierten Abstände.
Anders ausgedrückt: Im Gegensatz zum gerichteten (schwachen) Graph-Abstand ist $\sum_{e \in E_1}\delta_{(w)F}(e, s(e))$ nicht invariant unter dem Freiheitsgrad, den $s$
bei der Wahl der Bildpunkte in $G_2$, $s(u)$ und $s(v)$ hat.
\\
\\
TODO: Skizze
\\
\\
Erklärung zu der Skizze
Für einen Baum wie in Skizze xx. haben wir eine Vorauswahl von möglichen Knoten-Bildern unter einer unter einer $\varepsilon$-Platzierung zwischen den skizzierten Graphen getroffen.
Dabei soll der in blaue gezeichnete Baum auf den in rot gezeichneten Graphen abgebildet werden.
\\
\\
Um dies zu umgehen, werden wir für einen Knoten $u \in E_1$ sein Bild unter einer Graph-Zuordnung für jede seiner $\varepsilon$-Platzierung fixieren.
Als Fixpunkt wählen wir dafür einen Punkt in der $\varepsilon$-Platzierung mit minimalem Abstand zu $u$.

\subsection{Grundidee des Algorithmus}
Um nun unter diesen Einschränkungen an die Graph-Zuordnung den Min-Sum-Graph-Abstand$_{(w)F}(G_1,G_2)$ zu berechnen, wollen wir
eine $\varepsilon$-Platzierung $s: G_1 \to G_2$ konstruieren, welche das Optimalitätskriterium erfüllt.
\\
Anfangs werden wir die dafür notwendigen Berechnungsschritte motivieren.
Darauf aufbauend werden wir jeden Einzelschritt im Detail beschreiben sowie die Korrektheit und Laufzeit des Algorithmus argumentieren.
Abschließend demonstrieren wir den Algorithmus anhand eines konkreten Beispiels.
\\
Da $G_1$ ein Baum ist und $\varepsilon \leq \delta_{(w)F}(G_1, G_2)$ gilt,
ist die Existenz einer (schwachen) $\varepsilon$-Platzierung von $G_1$ nach $G_2$ garantiert (siehe dazu Akitaya et al. Seite 13, Lemma 6).
\\
\\
Dafür benötigen wir notwendigerweise zunächst alle $\varepsilon$-Platzierungen der Knoten in $G_1$.
Wir bezeichnen für $v \in E_1$ mit $P(v)$ die Menge aller $\varepsilon$-Platzierungen von $v$.
Für ein $p_v \in P(v)$ fixieren wir (ensprechend der zweiten Einschränkung in $\ref{Zweite Einschränkung}$) $s(v)$ für alle Graph-Zuordnungen,
die $v$ innerhalb von $p_v$ zuordnen. Wählen wir nun für jeden Knoten in $G_1$ eine beliebige aber feste $\varepsilon$-Platzierung, so definiert diese aufgrund der Fixpunkte
eine konkrete Abbildung $\tau: G_1 \to G_2$, die jedem Knoten in $G_1$ einen Punkt innerhalb seiner $\varepsilon$-Platzierung zuordnet.
\\
\\
Bemerkung:
$\tau$ ist im Allgemeinen keine Graph-Zuordnung und falls $\tau$ eine Graph-Zuordnung ist, so ist $\tau$ im Allgemeinen keine $\varepsilon$-Platzierung von $G_1$ nach $G_2$.
Denn für eine Kante $\{u, v\} \in E_1$ sind die $\varepsilon$-Platzierungen in $P(u)$ mit den $\varepsilon$-Platzierungen in $P(v)$ grundsätzlich nicht beliebig kombinerbar.
Ein $p_u \in P(u)$ und ein $p_v \in P(u)$ sind in $G_2$ im Allgemeinen nicht zwangsläufig verbunden und somit existiert nicht unbedingt ein einfacher Weg von $p_u$ nach $p_v$ in $G_2$.
\\
Sollte andernfalls ein einfacher Weg $w_{uv}$ in $G_2$ existieren, welcher $p_u$ und $p_v$ verbindet, so muss $w_{uv}$ keine (schwache) $\varepsilon$-Platzierung
von $\{u,v\}$ sein, d.h. wir können im Allgemeinen nicht annehmen, dass $\vec{\delta}_{(w)F}(\{u,v\}, w_{uv}) \leq \varepsilon$.
\\
\\
TODO: Beispiel 1
\\
\\
Entsprechend kommt für die Konstruktion einer $\varepsilon$-Platzierung $s: G_1 \to G_2$ für jede Kante $\{u,v\} \in E_1$ eine $\varepsilon$-Platzierungen $p_u \in P(u)$ von $u$ nur in Frage, wenn
es ein $p_v \in P(v)$ existiert, so dass $p_u$ und $p_v$ (schwach) voneinander erreichbar sind (gemäß Definition ...).
Dabei reicht es wiederum nicht, diese Bedingung jeweils nur für einzelne Kanten zu fordern.
\\
\\
TODO: Beispiel 2
\\
\\
In diesem Sinne wollen wir den Begriff der $\varepsilon$-Platzierung eines Knotens verschärfen.
\begin{Def}
	Gültige $\varepsilon$-Platzierung (engl. valid placement)
	Eine $\varepsilon$-Platzierung $p_v$ eines Knotens $v$ nennen wir (schwach) gültig, falls für jeden Nachbarn $u$ von $v$
	eine $\varepsilon$-Platzierung $p_u$ existiert, so dass $p_v$ und $p_u$ (schwach) voneinander erreichbar sind.
	Ansonsten nennen wir $p_v$ (schwach) ungültig.
\end{Def}

Wir wollen uns für die Konstruktion von $s$ lediglich auf die Auswahl von gültigen $\varepsilon$-Platzierungen beschränken.
Dafür löschen wir alle (schwach) ungültigen $\varepsilon$-Platzungen der Knoten von $G_1$.
\\
\\
Dabei gilt zu beachten, dass nach dem Löschen einer (schwach) ungültigen $\varepsilon$-Platzierung von $v$ aus $P(v)$
jede vor der Löschung noch (schwach) gültige $\varepsilon$-Platzierung eines Nachbarn von $v$ nun (schwach) ungültig sein kann.
Daher löschen wir rekursiv solange (schwach) ungültige $\varepsilon$-Platzierungen der Knoten in $G_1$, bis nach einer Löschung
keine (schwach) ungültige $\varepsilon$-Platzierung mehr existiert.
\\
\\
Existert nach der Bereinigung der (schwach) ungültigen $\varepsilon$-Platzierungen für jeden Knoten in $G_1$ noch mindestens eine
(schwach) gültige $\varepsilon$-Platzierung, so sichert uns ein Hilfssatz aus Akitaya et al. die Existenz und eine mögliche Konstruktion
für eine $\varepsilon$-Platzierung von $G_1$ nach $G_2$.

Hilfssatz:
Sei $G_1$ ein Baum und für $v \in V_1$ sei $\tilde{P}(v)$ die Menge der (schwach) gültigen $\varepsilon$-Platzierungenvon $v$ nach der
rekursiven Löschung aller (schwach) ungültigen $\varepsilon$-Platzierungen der Knoten in $G_1$. Enthält $\tilde{P}(v)$ für jeden Knoten
in $G_1$ mindestens eine (schwach) gültige $\varepsilon$-Platzierung, so hat $G_1$ eine schwache $\varepsilon$-Platzierung.


\subsection{Schritt 1}
Vorbereitung der ersten Einschränkung
\\
Zunächst initialisieren wir $\varepsilon$ mit $\varepsilon \geq \vec{\delta}_{(w)F}$.
\subsubsection{Schritt 2}
Berechnung der $\varepsilon$-Platzierungen und ihrer Fixpunkte (bezüglich \ref{Zweite Einschränkung}) für alle Knoten in $G_1$.

\subsubsection{Schritt 3}
Berechnung der Erreichbarkeit zwischen allen Knoten und ihren Nachbarn in $G_1$.

\subsubsection{Schritt 4}
Konstruktion von $s$, so dass sie das Optimalitätskriterium erfüllt, also dass für eine beliebige Graph-Zuordnung $\tilde{s}: G_1 \to G_2$ gilt:
$\sum_{e \in E_1}\delta_{(w)F}(e, s(e)) \leq \sum_{e \in E_1}\delta_{(w)F}(e, \tilde{s}(e))$.

Für eine $\varepsilon$-Platzierung eines Knotens in $G_1$ existieren grundsätzlich
Nehmen wir nun an, dass wir für eine $\varepsilon$-Platzierung eines Knotens in $G_1$ die Anzahl der möglichen
Das Kernproblem bei der Konstruktion einer Graph-Zuordnung, welche den Min-Sum-Graph-Abstands$_{(w)F}$ einhält ist, dass eine Kante $\{u,v\} \in E_1$
\\
um $ \min_{s: G_1 \to G_2} \sum_{e \in E_1} \delta_{(w)F}(e, s(e))$ zu berechnen, werden wir iterativ eine Graph-Zuordnung $\tilde{s}: G_1: G_2$ konstruieren,
welche das Optimalitätskriterium, dass für alle Kanten für einen Teilgraph $G_1$ und eine Auswahl von $\varepsilon$-Kantenplatzierung einhält.
$\varepsilon$ ist dabei mindestens so groß wie der gerichtete (schwache) Graph-Abstand von $G_1$ und $G_2$.
\\
Da $G_1$ ein Baum ist, können wir diese Zuordnungen jedoch so an den Knoten durchführen, dass ein Teil von $G_1$ stets optimal zugeordnet ist.

\section{Ausblick}
In dieser Arbeit haben wir uns, motiviert durch die Eigenschaften des (schwachen) gerichteten Graph-Abstands, mit dem
Min-Sum-Graph-Abstand$_{(w)F}$ beschäftigt. Dabei haben wir die Frage der NP-Schwerheit des Min-Sum-Graph-Abstands$_{(w)F}$ für allgemeine
Graphen geklärt und einen Polynomialzeit-Algorithmus zur Berechnung von Min-Sum-Graph-Abstand$_{(w)F}(G_1, G_2)$ beschrieben, sofern $G_1$ ein Baum ist
und wir die Menge der Graph-Zuordnungen (wie in \ref{Einschränkungen} beschrieben) einschränken.

Buchin et al. gehen des Weiteren davon aus, dass sich der Min-Sum-Graph-Abstand$_{(w)F}$ auch auf planaren Graphen und
auf Bäumen ohne die notwendige Einschränkung (aus \ref{Zweite Einschränkung}) ebenfalls NP-Schwer ist.
Formale Beweise dieser Annahmen stehen dabei noch aus.
\\
\\
Allgemein beschreibt der Min-Sum-Graph-Abstand$_{dist}$ eine ganze Klasse von Abstandsmaßen auf Graphen, deren Mitglieder sich durch die Wahl
eines geeigneten Abstandsmaßes für die Messung der Kanten-Zuordnungen unterscheiden.
Wir haben bei unserer Betrachtung des diskreten Fréchet-Abstands ein weiteres Mitglied dieser oben beschriebenen Klasse kennengelernt.
Dabei gehen wir allerdings davon aus, nur einen kleinen Blick über den Tellerrand der noch offenen Möglichkeiten gewagt zu haben.
\newpage\null\thispagestyle{empty}\newpage
%   Eventuell eine Leerseite vor der Literaturangabe, falls diese sonst auf einer Rückseite angegeben würde. (Seitennummer muss ungerade sein!)

\begin{thebibliography}{4}
	\bibitem{Buchin}
		Maike Buchin, Bernhard Kilgus. Distance Measures for Embedded Graphs - Optimal Graph Mappings.
		\textit{European Workshop on Computational Gemoetry,} 2020.

	\bibitem{Akitaya}
		Hugo Akitaya, Maike Buchin, Bernhard Kilgus, Stef Sijben, Carola Wenk. Distance measures for embedded graphs
		\textit{Computational Geometry,} Volume 95, 2021.

	\bibitem{Alt}
		Hemlut Alt, Michael Godau. Computing the Fréchet distance between two polygonal curves.
		\textit{Int. Journal of Computational Geometry and Applications,} 5:75-91, 1995.

	\bibitem{Rote}
		Günter Rote, Lexicographic Fréchet distance.
		\textit{European Workshop on Computational Gemoetry,} 2014.

\end{thebibliography}

% Symbolverzeichnis definieren
%\nomenclature{$c$}{Speed of light in a vacuum inertial frame}
%\nomenclature{$h$}{Planck constant}

%\printnomenclature
\end{document}
