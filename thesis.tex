% Notwendige Start-Codezeile
\documentclass[a4paper, 12pt, twoside]{article}

% Präambel

% Pakete
\usepackage[left=3cm,right=2.5cm,bottom=3.5cm,top=2.5cm]{geometry} % Ok die Seitenrändereinstellungen passen so.
\usepackage{amsfonts}
\usepackage[utf8]{inputenc} % Kodierung
\usepackage[ngerman]{babel} % Sprache
\usepackage{amssymb}        % Mathematische Symbole wie z.b ganze Zahlen, reelle Zahlen etc.
\usepackage{amsmath}        % Um diverse mathematische Symbole nutzen zu können.
\usepackage{amsthm}         % Um Definitionen, Theoreme, Bemerkungen, Beispiele machen zu können.
\usepackage{mathtools}      % Dieses Paket liefert nützliche Werkzeuge, z.B "defined as equal" - sign uvm.
\usepackage{enumitem}       % Um individuelle Listen zu bauen.
\usepackage{xcolor}         % Um farbigen Text machen zu können. Diesen kann ich nutzen um wichtige persönliche Notizen hervorzuheben.
\usepackage{fancyhdr}       % Um saubere Kopf und Fußzeilen sowie Seitenzahlen zu erzeugen.
\usepackage{setspace}       % Damit kann ich Leerzeilen im Dokument einfügen.
\usepackage{graphicx}       % Einbinden von externen Bildern
% insbesondere beim Inhaltsverzeichnis nötig, da dieses sonst zu
% gequetscht aussieht.


% Symbolverzeichnis
%\usepackage{nomencl}
%\makenomenclature
%\renewcommand{\nomname}{Symbolverzeichnis}


% Globale Festlegungen
\setlength{\topsep}{4ex plus0.5ex minus0.5ex} % Festlegung: Größe der Absätze nach Definition, Theorem, etc.
\setlength\parindent{0pt} % Festlegung: Kein Einschub nach rechts.
\setstretch{1.2} % Festlegung: Zeilenabstand ist 1.2 statt 1.
\raggedbottom % twoside sorgt dafür, dass der Platz, der durch \\ und einer Leerzeile erzeugt wird sehr groß ist und nicht nur eine Leerzeile, was ich nämlich erzielen möchte.
              % Dieses Kommando sorgt dafür, dass dies wiederhergestellt wird und twoside trotzdem wirkt.

% Vorlagen
% 1) [before = \leavevmode\vspace{-\baselineskip}]
% // Um bei Theoremen mit Listen zu starten.

% Eigene Befehle

\newcommand\logeq{\mathrel{\vcentcolon\Longleftrightarrow}} % Dieser Befehl realisiert mittels "\logeq" ein "defnierendes Äquivalenzzeichen".
\newcommand{\ts}{\thinspace} % Damit ich nicht so viel schreiben muss, wenn ich kleine Leerzeilen hinzufügen möchte. Was ich oft tue bei Mengendefinitionen, da mir der Platz dort zu klein ist.

% Eigene Theoremstyles
% Format 1
\newtheoremstyle{Format1}
{\topsep}   % ABOVESPACE
{\topsep}   % BELOWSPACE
{\normalfont}  % BODYFONT
{0pt}       % INDENT (empty value is the same as 0pt)  % Das rückt den Kopf nach rechts ein
{\bfseries} % HEADFONT
{\newline}  % HEADPUNCT
{5pt plus 1pt minus 1pt} % HEADSPACE
{}          % CUSTOM-HEAD-SPEC


% Eigene Theorem-Umgebungen
\theoremstyle{Format1} % Alle Theoremumgebungen hierunter folgen den Spezifikationen vom "Format 1" - Theoremstil
\newtheorem{Def}{Definition}[section]       % Definition
\newtheorem*{Definition}{Definition}        % Definition (unnummeriert)
\newtheorem{Bsp}[Def]{Beispiel}             % Beispiel
\newtheorem{Bem}[Def]{Bemerkung}            % Bemerkung
\newtheorem{Satz}[Def]{Satz}                % Satz
\newtheorem*{Bez}{Bezeichnung}              % Bezeichnung (unnummeriert)
\newtheorem{Folg}[Def]{Folgerung}           % Folgerung
\newtheorem*{Folgerung}{Folgerung}          % Folgerung (unnummeriert)
\newtheorem{Lem}[Def]{Lemma}                % Lemma
\newtheorem*{Grundmodell}{Grundmodell}
\newtheorem*{Aussage}{Aussage}
\newtheorem*{Herleitung}{Herleitung}

% Code aus Stackexchange, welcher bei Nutzen von Norm oder Betrag automatisch passende Betragsgrößen generiert, also die Beträge dem Ausdruck entsprechend vergrößert:

\DeclarePairedDelimiter\abs{\lvert}{\rvert}%
\DeclarePairedDelimiter\norm{\lVert}{\rVert}%

% Swap the definition of \abs* and \norm*, so that \abs
% and \norm resizes the size of the brackets, and the
% starred version does not.
\makeatletter
\let\oldabs\abs
\def\abs{\@ifstar{\oldabs}{\oldabs*}}
%
\let\oldnorm\norm
\def\norm{\@ifstar{\oldnorm}{\oldnorm*}}
\makeatother

% Code aus Stackexchange: Sorgt dafür, dass die Abstände zwischen Zeilen in Allign Umgebungen etwas größer sind, also normal. Die Standardabstände sind mir zu gering.

\addtolength{\jot}{0.5em}


% Beginn des Dokumentes
\begin{document}

\newgeometry{} % Damit die Titelseite nicht von den Einstellungen des geometry packages beeinflusst wird.
% Titelblatt der Arbeit
\begin{titlepage}
	\begin{center}
		\vspace*{1cm}

		\Huge
		\textbf{Bachelorarbeit}

		\vspace{0.5cm}
		\LARGE
		Optimale Zuordnungen auf Geometrischen Graphen

		\vspace{1.5cm}
		\large Sebastian Koletzko\\

		\vspace{1cm}
		Datum: 17.12.23

		\vfill

		\vspace{5cm}

		%\includegraphics[width=0.4\textwidth]{RuhrU}

		\large
		Fakultät für Mathematik
		\\
		Ruhr-Universität Bochum
		\\
		\vspace{0.5cm}
		Prof. Dr. Maike Buchin
		\\
		Dr. Daniela Kacso

	\end{center}
\end{titlepage}

\restoregeometry % Damit die Titelseite nicht vom geoemtry-package beeinflusst wird - Endcodezeile.

\newpage\null\thispagestyle{empty}\newpage % Nach der Titelseite kommt immer eine leere Seite, denn die nachfolgende Seite, ist die Rückseite der Titelseite. Und ich möchte nicht auf der Rückseite direkt anfangen.

\thispagestyle{empty} % Damit keine Seitenzahl auf der Erklärung-Seite auftaucht.

\textbf{Eigenständigkeitserklärung:}
\\
Hiermit erkläre ich, dass ich die heute eingereichte Bachelorarbeit selbstständig verfasst und keine anderen als die angegebenen Quellen und Hilfsmittel benutzt sowie Zitate kenntlich gemacht habe.
Bei der vorliegenden Bachelorarbeit handelt es sich um in Wort und Bild völlig übereinstimmende Exemplare. Ich erkläre weiterhin, dass die vorliegende Arbeit noch nicht im Rahmen eines anderen Prüfungsverfahrens eingereicht wurde.
\\
\\
Witten, den 17.12.23
\\
\\
Sebastian Koletzko


% Inhaltsverzeichnis. (Dieses wird nach Sektionen geordnet)
\newpage
\tableofcontents
\newpage\null\thispagestyle{empty}\newpage % Eine Leerseite nach dem Inhaltsverzeichnis.
\section{Einleitung}

Die geometrische Graphentheorie beschäftigt sich unter Anderen mit Graphen, deren Knoten und Kanten durch geometrische Objekte repräsentiert werden.
Wir bezeichnen solche Graphen als eingebettete Graphen. So bilden die bekannten planaren Graphen z.B. die Teilmenge von in den euklididschen Raum eingebetteten Graphen, deren Kanten sich als ebene Kurven repräsentieren lassen, dass
sich je zwei Kanten nur an ihren inzidenten Knoten schneiden.
\\
In dieser Arbeit wollen wir uns insbesondere mit den sogenannten geometrischen Graphen beschäftigen. Hierbei handelt es sich um Graphen, die unter bestimmten Vorraussetzungen in die euklidische Ebene eingebettet werden.
Dabei werden die Knoten durch Punkte in der Ebene und die Kanten durch Geradenstücke zwischen den inzidenten Knotenpunkten repräsentiert.
Geometrische Graphen eignen sich insbesondere zur Repräsentation von Netzwerken (wie z.B. Straßennetzen).
\\
Diesbezüglich wollen wir verschiedene Repräsentationen eines Netzwerks miteinander vergleichen, was im Umkehrschluss bedeutet, dass wir geometrische Graphen miteinander vergleichen.
Wir interessieren uns dabei für die Ähnlichkeit zweier Graphen. Zum Vergleich zweier Graphen bzw. zwei geometrischer Objekte stoßen wir unweigerlich auf den Begriff eines Abstandsmaßes.
Unter einem Abstandsmaß verstehen im Allgemeinen eine reellwertige Zuordung zum Vergleich zweier geometrischer Objekte wie z.B. zum Vergleich von zwei Kurven im euklidischen Raum.
Für den Vergleich von eingebetteten Graphen existieren dabei bereits einige bekannte Abstandsmaße.
\\
Eine zentrale Fragestellung dabei ist, wie man neue Abstandsmaße auf eingebetteten Graphen definiert und unter welchen Bedingungen sich ein solches Abstandsmaß effizient berechnen lässt.
\\
Ferner interessieren wir uns für die Eigenschaften eines Abstandsmaß und seiner geometrischen Bedeutung im Bezug auf die zu vergleichenden Objekte.
Einige Abstandsmaße erfassen z.B. beim Vergleich von eingebetteten Graphen lediglich die Repräsentation der Graphen im Sinne ihrer Punktmenge im euklidischen Raum berücksichtigen dabei nicht die ihnen
durch ihre abstrakte Beschreibung ausgewiesene Konnektivität.
\\
Um die Ählichkeit zweier eingebetter Graphen insbesondere in Bezug auf die Konnektivität des Graphen zu beschreiben, haben Akitaya et al. \cite{Akitaya} neue Abstandsmaße eingeführt und ihre Eigenschaften untersucht.
\\
Besonderheit bei diesen Abstandmaßen ist, dass sie auf einer stetigen Zuordnung zwischen den zu vergleichenden Graphen basieren.
\\
Vor dem Hintergrund, dass die "directed (weak) graph distance" eine bottle-neck distance beschreibt, wird bei der min-sum graph distance das Ziel verfolgt, lokale Abstände zu berücksichtigen.
Darauf aufbauend haben \cite{Buchin} Buchin et al. ein weiteres Abstandsmaß definiert, welches mögliche lokale Optimierungen der Graph-Zuordnung berücksichtigt.
[Ein kleiner Text über den Stand der Wissenschaft]

Mit diesem Abstandsmaß wollen wir uns in dieser Arbeit hauptsächlich beschäftigen.
In Kapitel 2 fassen wir dazu zunächst die Notation und die zentralen Begriffe aus [2] zusammen, welche den Grundstein für die Auseinandersetzung mit dem neuen Abstandsmaß bilden.
In Kapitel 3 werden wir den Min-Sum-Graph-Abstand motivieren und einführen. Anschließend werden wir uns mit den Bedingungen und einem Ansatz beschäftigen, um den Min-Sum-Graph-Abstand
unter Verwendung des Fréchet-Abstands in Polynomialzeit zu berechnen.
Abschließend beweisen wir in Kapitel 4 die NP-Schwerheit des Min-Sum-Graph-Abstands unter Verwendung des Fréchet-Abstands für allgemeine Graphen.

Zur Berechnung des Min-Sum-Graph-Abstands haben Buchin et al in [1] einen Algorithmus beschrieben, welcher unter gewissen Einschränkungen in Polynomialzeit läuft.
Bevor wir uns abschließend diesem Algorithmus widmen, werden wir die NP-Schwerheit des Min-Sum-Graph-Abstands für allgemeine Graphen beweisen.
\newpage

\section{Grundlagen}
Wir betrachten hier den $ \mathbb{R}^n $ grundsätzlich als metrischen Raum, versehen mit der euklidischen Norm.
\begin{Def}
	Ein \textit{Weg} in $\mathbb{R}^n$ st eine stetige Abbildung $ \phi: [0,1] \to \mathbb{R}^n $ von einem geschlossenen Intervall $[0,1]$ nach $\mathbb{R}^n$.
	\\
	Das Bild von $I$ unter $\phi$ nennen wir \textit{Kurve}.
	\\
	Wir bezeichnen $\phi(0)$ als den \textit{Startpunkt} und $\phi(1)$ als den \textit{Endpunkt} des Weges und bezeichnen die Menge
	$\{\phi(0), \phi(1)\}$ auch als Randpunkte des Weges. Sofern nicht ausdrücklich erwähnt, fordern wir dabei stets $\phi(a) \neq \phi(b)$.
	\\
	Ein \textit{einfacher Weg} ist ein Weg, welcher injektiv auf $[0,1]$ ist.
\end{Def}

\begin{Def}
	Sei $G=(V,E)$ ein endlicher, ungerichteter Graph.

	Unter einem \textit{eingebetteten Graphen} verstehen wir einen Graphen $G$
	zusammen mit seiner \textit{Einbettung} $\omega: G \to \mathbb{R}^n$, welche
	jedem Knoten $v \in V$ einen eindeutigen Punkt $v_0 \in \mathbb{R}^n$ zuordnet und jeder Kante $\{u,v\} \in E$
	einen einfachen Weg $\phi: [0,1]: \to \mathbb{R}^n$ mit den Randpunkten $\omega(u)$ und $\omega(v)$ zuordnet.
	\\
	Sind alle Kanten in $\mathbb{R}^n$ als Geradenstücke mit den zur jeweiligen Kante inzidenten Knoten als Randpunkte realisiert, so
	nennen wir $G$ einen \textit{geradlinig eingebetteten Graphen}.
	\\
	Dabei dürfen sich Kanten von $G$ grundsätzlich in einem Punkt schneiden.
	\\
	\\
	Für einen eingebetteten Knoten $v \in V$ definieren wir seinen \textit{$\varepsilon$-Ball} $B_{\varepsilon}(v)$ als die Menge
	$\{x \in \mathbb{R}^n.: \|v-x\| \leq \varepsilon\}$
	und für eine (geradlinig) eingebette Kante $e \in E$ definieren wir ihren \textit{$\varepsilon$-Schlauch} $T_{\varepsilon}(e)$ als die Menge
	$\{x \in \mathbb{R}^n: \min_{a \in e}\|a-x\| \leq \varepsilon\}$.
	\\
	\\
	Einen in die euklidische Ebene geradlinig eingebetteten Graphen bezeichnen wir als \textit{geometrischen Graphen}.
\end{Def}

\begin{Def}
	Einfacher Weg in einem Graphen
	\\
	Sei $G$ ein geradlinig eingebetteter Graph.
	\\
	Ein einfacher Weg in $G$ ist eine stetige, injektive Abbildung $\tilde{\phi}: [0,1] \to G$.
	Dabei betrachten bezüglich der Stetigkeit $G$ als Topologischen Graphen (TODO: Referenz).
	\\
	Die Kurve von $\tilde{\phi}$ repräsentiert in gewisser Weise eine zusammenhängende Komponente von $G$ bestehend aus
	Kanten und Teilkanten. Dabei ist es jedoch nicht erlaubt, dass $\tilde{\phi}$ Umwege über Kantenschnitte in $G$ nimmt.
	Wir fordern also insbesondere, dass $\tilde{\phi}$ die abstrakte Konnektivität von $G$ berücksichtigt.
	\\
	Wenn wir im weiteren Verlauf von einem einfachen Weg sprechen, so beziehen wir die damit verbundene Kurve als Bild des Weges
	mit ein und benutzen die beiden Begriffe mitunter synonym. Dies soll hervorheben, dass die eigentliche Parametrisierung
	der Kurve nicht von entscheidender Bedeutung ist. Entscheidend für den weiteren Verlauf ist für uns nur die Vorraussetzung, dass die
	Parametrisierung stetig und injektiv ist.
	\\
	Für einen Weg $W$ in einem einegebetteten Graphen definieren wir seine \textit{Länge} als die Anzahl der Geradenstücke, aus
	denen sich $W$ zusammensetzt. Wir kennzeichnen die Länge von $W$ durch $|W|$.
\end{Def}

\subsection{Abstände auf geometrischen Graphen}

Seien $ G_1=(V_1, E_1) $ und $ G_2=(V_2, E_2) $ für den gesamten weiteren Verlauf geometrische Graphen mit
$n_1 = |V_1|, m_1 = |E_1|, n_2 = |V_2|$ und $m_2 = |E_2|$.
Im weiteren Verlauf werden wir für einen Graphen nicht explizit zwischen seiner abstrakten Beschreibung und seiner Einbettung unterscheiden.

\begin{Def} \label{Definition Graph-Zuordnung}
	Graph-Zuordnung (engl. graph mapping)
	Wir nennen eine Abbildung $s: G_1 \to G_2 $ eine \textit{Graph-Zuordnung}, falls
    	\begin{enumerate}
		\item[1)] s jeden Knoten $ v \in V_1 $ auf einen Punkt innerhalb von $ G_2 $ abbildet und
		\item[2)] s alle Kanten $ \{u,v\} \in E_1 $ auf einen einfachen Weg in $G_2$ abbildet, wobei $s(u), s(v)$ die Randpunkte des Weges sind.
    	\end{enumerate}

	Eine Graph-Zuordnung definiert somit eine stetige Abbildung von $ G_1 $ nach $ G_2 $ und ist im Allgemeinen weder injektiv noch surjektiv.
	In gewisser Weise könnte man hier einen Bezug zum Begriff der \textit{Einbettung} aus der Topologie herstellen. (TODO: Referenz).
	Hier ist es jedoch so, dass wir für die Existenz einer Graph-Zuordnung von $ G_1 $ nach $ G_2 $ nicht so strenge Bedingungen stellen, insbesondere
	müssen $G_1$ und $G_2$ nicht homöomorph sein.
\end{Def}

\begin{Def} \label{Definition Fréchet-Abstand}
	Fréchet-Abstand (engl. fréchet distance)
	Für zwei Wege $ f, g: [0,1] \to \mathbb{R}^n $ definieren wir den \textit{Fréchet-Abstand} von $f$ und $g$ als
	$$ \delta_F(f,g) =  \inf_{\sigma:[0,1] \to [0,1]} \; \max_{t \in [0,1]} \lVert f(t)-(g(\sigma(t)) \rVert, $$
	wobei sich $\sigma $ über alle orientierungserhaltenen Homöomorphismen erstreckt.
	\\
	\\
	Wir definieren den \textit{schwachen Fréchet-Abstand} von $f$ und $g$ als
	$$\delta_{wF}(f,g) =\inf_{\alpha , \beta :[0,1] \to [0,1]} \; \max_{t \in [0,1]} \lVert f(\alpha(t))-(g(\sigma(t)) \rVert,$$
	wobei sich $\alpha$ und $\beta$ über alle stetigen Abbildungen erstrecken, welche die Randpunkte fixieren, also $\alpha(0) = \beta(0) = 0$
	und $\alpha(1) = \beta(1) = 1$.
	\\
	\\
	Die Klammer-Notation, z.B. $ \delta_{(w)F} $, benutzen wir zugunsten der Kompaktheit nachfolgend, wenn wir im jeweiligen Kontext den Fréchet-Abstand sowie den schwachen Fréchet-Abstand zugleich -
	wir schreiben in diesem Fall auch \textit{(schwacher) Fréchet-Abstand} - adressieren wollen.
	\\
	\\
	Zur Anschauung des Fréchet-Abstands stelle man sich $f$ und $g$ als Kurven im $\mathbb{R}^2$ vor. Der (euklidische) Abstand zwischen $f$ und $g$ zum Zeitpunkt $t_0 \in [0,1]$ entspricht gerade der Länge der Strecke
	mit den Randpunkten $f(t_0)$ und $g(t_0)$. Bei dem Fréchet-Abstand dürfen wir eine Kurve reparametrisieren, um den maximal angenommenen euklidischen Abstand zwischen $f$ und $g$ zu minimieren.
	TODO: Anschauung ergänzen
	\\
	Für den Fréchet-Abstand darf einer von beiden auf seinem festgelegten Weg - welcher der Spur der Kurve entspricht - zu jedem Zeitpunkt beliebig beschleunigen oder sogar stehenbleiben, aber nicht die Richtung wechseln.
	Dies mit dem Ziel, die dabei maximal angenommene Länge der Leine möglichst zu minimieren.
	Für den schwachen Fréchet-Abstand dürfen der Mann sowie sein Hund auf ihrem Weg beliebig beschleunigen und auch die Richtung ändern.
	\\
	Wir bezeichnen einen solchen maximal-angenommen Abstand zwischen zwei Objekten sinngemäß auch als \textit{Flaschenhals-Abstand}.
	\\
\end{Def}

\begin{Def} \label{Definition Graph-Abstand}
	Abstände auf Graphen
	Wir definieren den \textit{gerichteten Graph-Abstand} $ \vec{\delta}_G $ als
	$$ \vec{\delta}_G(G_1,G_2) = \inf_{s: G_1 \to G_2} \: \max_{e \in E_1} \delta_F(e, s(e)) $$
	und den \textit{gerichteten schwachen Graph-Abstand} $ \vec{\delta}_{wG} $ als
	$$  \vec{\delta}_{wG}(G_1,G_2) = \inf_{s: G_1 \to G_2} \: \max_{e \in E_1} \vec{\delta}_{wF}(e, s(e)), $$
	wobei s sich über alle Graph-Zuordnungen erstreckt.
\end{Def}

Im Vergleich zu anderen Abstandsmaßen (TODO: Referenz hinzufügen), welche primär die geometrische Ähnlichkeit von $G_1$ und $G_2$ berücksichtigen, hat neben den geometrischen Eigenschaften auch die (abstrakte) Konnektivtät von $G_1$ und $G_2$
Implikationen für das Ausmaß des gerichteten (schwachen) Graph-Abstands.
\\
\\
TODO: Beispiel/Erklärung
\\
Ähnlich wie der (schwache) Fréchet-Abstand beschreibt auch der gerichtete (schwache) Graphabstand einen Flaschenhals-Abstand.
Generell können viele Graph-Zuordnungen $s: G_1 \to G_2$ existieren, die den gerichteten (schwachen) Graph-Abstand einhalten.

\subsection{Lokale Optimierungen}

Im Anwendungsfall kann man sich z.B. mehrere Rekonstruktionen $G_{1,1}, ..., G_{1,n}$ eines Netzwerkes $G_2$ vorstellen.
Anschließend stelle man sich nun die Frage, welche der Rekonstruktionen dem urpsprünglichen Netzwerk $G_2$ am ähnlichsten ist bzw. dieses entsprechend eines konkreten Abstandsmaßes
für Graphen am besten.
Der gerichtete (schwache) Graph-Abstand zwischen einem $G_{1i}$ $(i \in \{1,...,n\})$ und $G_2$ beschreibt dabei lediglich eine obere Schranke für die größte Abweichung zwischen
einer Rekonstruktion $G_{1i}$ und $G_2$.
Unter Umständen haben evlt. sogar alle $G_{1i}$ denselben Flaschenhals-Abstand zu $G_2$, da sie dieselbe entscheidende Abweichung bei der Rekonstruktion von $G_2$ erzeugen.
Insbendere interessieren wir uns hier nun für die lokale Güte der Rekonstruktion, welche für den gerichteten (schwachen) Graph-Abstand keine Rolle spielt.
\\
\\
Dies motiviert die Einführung eines weiteren Abstandsmaßes auf Graphen, welches auch lokale Optimierungen bei der Zuordnungen der Kanten in Betracht zieht.
\\
\\
Zuvor merken wir an, dass im Allgemeinen eine Graph-Zuordnung $\tilde{s}: G_1 \to G_2$, die jede Kante $e \in E_1$ optimal bezüglich des (schwachen)
Fréchet-Abstands auf $G_2$ abbildet, so dass $\delta_{(w)F}(e, \tilde{s}(e)) \leq \delta_{(w)F}(e, s(e))$, nicht existiert (siehe dazu [2], Seite xx).
\\
\\
Als Alternative bietet es sich an, die summierten Abstände zwischen den Kanten in $G_1$ und ihren Bildern in $G_2$ unter der Graph-Zuordnung zu minimieren.

\subsection{Optimale Graph-Zuordnungen}
Damit widmen wir uns nun einem Abstandsmaß, welches für uns im weiteren Verlauf das Optimalitätskriterum - bezüglich lokaler Abstände zwischen Kanten und ihren Bildern - für Graph-Zuordnungen beschreibt.

\begin{Def} \label{Definition Min-Sum}
	Min-Sum-Graph-Abstand (engl. min-sum graph distance)

	Für die Graphen $G_1$ und $G_2$ definieren wir allgemein den \textit{Min-Sum-Graph-Abstand}$_{dist}(G_1, G_2)$ als
	$$\min_{s: G_1 \to G_2} \sum_{e \in E_1} dist(e, s(e)),$$
	wobei sich $s$ über alle Graph-Zuordnungen von $G_1$ nach $G_2$ erstreckt und $dist$ ein unspezifisches Abstandsmaß für den Vergleich von Kurven im euklidischen Raum repräsentiert.
	\\
	\\
	Wählen wir für $dist$ den (schwachen) Fréchet-Abstand als zugrundeliegendes Abstandsmaß, so bezeichnen wir entsprechend
	\textit{Min-Sum}$_{(w)F}(G_1, G_2)$ als $$\min_{s: G_1 \to G_2} \sum_{e \in E_1} \delta_{(w)F}(e, s(e)).$$
	\\
	\\
	Ferner bezeichnen wir mit $Sum_{(w)F}(s) = \sum_{e \in E_1}\delta_{(w)F}(e, s(e))$ die Summe (schwachen) Fréchet-Abstände zwischen den Kanten
	in $G_1$ und ihren durch $s$ zugeordneten Wegen in $G_2$.
	\\
	Wir nennen $s$ eine Min-Sum$_{(w)F}$ Zuordnung, falls
	\\
	$Sum_{(w)F}(s) = \text{Min-Sum-Graph-Abstand}_{(w)F}(G_1,G_2)$.

\end{Def}

\section{Ein polynomieller Algorithmus}

Im Nachfolgenden werden wir zur Berechnung des Min-Sum-Graph-Abstands$_{(w)F}$ den Algorithmus aus [1] beschreiben und diesen anhand eines konkreten Beispiels ausführen.
Sei $G_1$ von nun an ein Baum, also ein kreisfreier, zusammenhängender Graph.

\subsection{Einschränkungen an die Graph-Zuordnungen} \label{Einschränkungen}
In der Definition des Min-Sum-Graph-Abstands$_{(w)F}(G_1, G_2)$ ziehen wir alle Graph-Zuordnungen $s: G_1 \to G_2$ in Betracht.
Im weiteren Verlauf werden wir diese Menge mit zwei Einschränkungen versehen.
Die erste Einschränkung ist dabei optional. Die zweite Einschränkung wird notwendig sein, um die polynomielle Laufzeit des Algorithmus zu gewährleisten.

\begin{Def}
	$\varepsilon$-Platzierung (engl. $\varepsilon$-Placement)
	Eine \textit{$\varepsilon$-Platzierung eines Knotens $v \in V_1$} ist eine maximal zusammenhängende Komponente von $G_2$ eingeschränkt auf $B_{\varepsilon}(v)$.
	Eine \textit{$\varepsilon$-Platzierung einer Kante $e = \{u,v\} \in E_1$} ist ein Weg $W$ in $G_2$, welcher eine Platzierung $p_u$ von $u$ mit einer Platzierung $p_v$ von $v$
	so verbindet, dass $\delta_F(e, W) \leq \varepsilon$.
	\\
	In diesem Fall nennen wir $p_u$ und $p_v$ \textit{voneinander erreichbar}.
	\\
	Eine \textit{$\varepsilon$-Platzierung von $G_1$} ist eine Graph-Zuordnung $s: G_1 \to G_2$, so dass $s$ jede Kante $e \in G_1$ auf eine $\varepsilon$-Platzierung von $e$ abbildet.
	\\
	Eine schwache $\varepsilon$-Platzierung einer Kante $e = \{u,v\} \in E_1$ ist ein Weg $W$ in $G_2$, welcher eine Platzierung $p_u$ von $u$ mit einer Platzierung $p_v$ von $v$
	so verbindet, dass $\delta_{wF}(e, W) \leq \varepsilon$.
	\\
	In diesem Fall nennen wir $p_u$ und $p_v$ \textit{schwach voneinander erreichbar}.
	\\
	Eine schwache \textit{$\varepsilon$-Platzierung von $G_1$} ist eine Graph-Zuordnung $s: G_1 \to G_2$, so dass $s$ jede Kante $e \in G_1$ auf eine schwache $\varepsilon$-Platzierung von $e$ abbildet.
	\\
	\\
	Als Abkürzung benutzen wir auch auch die Begriffe \textit{Knoten-Platzierung} und \textit{Kanten-Platzierung}, sofern im Kontext dadurch kein Bezug verloren geht.
\end{Def}

\subsubsection{Erste Einschränkung}

Wir haben den Begriff des Min-Sum-Graph-Abstands primär vor dem Hintergrund motiviert, optimale Graph-Zuordnungen auf geometrischen Graphen zu betrachten,
die bereits einen Flaschenhals-Abstand wie den gerichteten (schwachen) Graph-Abstand einhalten.

Um diesen Zusammenhang zu erhalten, werden wir die in Betrachtung zu ziehenden Graph-Zuordnungen zunächst auf die Menge
$S = \{s: G_1 \to G_2 \text{ $|$ } s$ ist eine (schwache) $\varepsilon$-Platzierung von $G_1$ nach $G_2\}$ einschränken.
\\
Generell können wir auch $\varepsilon \geq \vec{\delta}_{(w)F}(G_1, G_2)$ fordern, für größere $\varepsilon$ erhöht sich entsprechend der Freiheitsgrad in der Auswahl
der möglichen Graph-Zuuordnungen und damit verrringert sich auch potentiell der unter dieser Einschränkung bewertete Min-Sum$_{(w)F}$ Abstand.
\\
Wir merken zusätzlich an, dass eine $\varepsilon$-Platzierung von $G_1$ nach $G_2$ für $\varepsilon < \vec{\delta}_{(w)F}(G_1,G_2)$
nicht existiert. Dies folgt direkt aus den Definitionen \ref{Definition Graph-Abstand} und \ref{Definition Min-Sum}.

\subsubsection{Zweite Einschränkung} \label {Zweite Einschränkung}
Da wir hier zur Berechnung der Abstände zwischen einer Kante $\{u,v\} \in E_1$ und ihrem Bild $s(\{u,v\})$ unter einer Graph-Zuordnung $s$ den (schwachen) Fréchet-Abstand verwenden,
hat die Wahl der Bilder $s(u)$ und $s(v)$, innerhalb der entsprechenden $\varepsilon$-Platzierungen von $u$ und $v$, eine direkte Auswirkung auf die summierten Abstände.
Anders ausgedrückt: Im Gegensatz zum gerichteten (schwachen) Graph-Abstand ist $\sum_{e \in E_1}\delta_{(w)F}(e, s(e))$ nicht invariant unter dem Freiheitsgrad, den $s$
bei der Wahl der Bildpunkte in $G_2$, $s(u)$ und $s(v)$ hat.
\\
\\
TODO: Skizze
\\
\\
Erklärung zu der Skizze
Für einen Baum wie in Skizze xx. haben wir eine Vorauswahl von möglichen Knoten-Bildern unter einer unter einer $\varepsilon$-Platzierung zwischen den skizzierten Graphen getroffen.
Dabei soll der in blaue gezeichnete Baum auf den in rot gezeichneten Graphen abgebildet werden.
\\
\\
Um dies zu umgehen, werden wir für einen Knoten $u \in E_1$ sein Bild unter einer Graph-Zuordnung für jede seiner $\varepsilon$-Platzierung fixieren.
Als Fixpunkt wählen wir dafür einen Punkt in der $\varepsilon$-Platzierung mit minimalem Abstand zu $u$.

\subsection{Ansätze zur Konstruktion eines polynomiellen Algorithmus} \label{Grundidee des Algorithmus}
Um nun unter diesen Einschränkungen an die Graph-Zuordnungen von $G_1$ nach $G_2$ Min-Sum$_{(w)F}(G_1,G_2)$ zu berechnen, konstruieren
eine $\varepsilon$-Platzierung $s: G_1 \to G_2$, die das Optimalitätskriterium erfüllt.
Für eine solche Abbildung $s$ gilt dann entsprechend $\sum_{e \in E_1}\delta_{(w)F}(e, s(e) = \text{Min-Sum}_{(w)F}(G_1,G_2)$.
\\
\\
Der algorithmische Ansatz um das Entscheidungsproblem, ob eine $\varepsilon$-Platzierung von $G_1$ existiert (siehe [2] Seiten 9-11), liefert das
Grundgerüst für die Konstruktion von $s$. Wir wollen den algorithmischen Ansatz motivieren und die Berechnungsschritte im Detail beschreiben.
Der Algorithmus zerfällt dabei in insgesamt 4 Einzelschritte, die wir im Folgenden entsprechend kennzeichnen werden.

\subsubsection{$\varepsilon$-Platzierungen der Knoten} \label{Platzierungen der Knoten}
Die Zuordnung eines Knotens $u \in V_1$ auf einen Punkt innerhalb seiner $\varepsilon$-Platzierung impliziert stets die Randpunkte $w(0)$, $w(1)$
für einen einfachen Weg $w: [0,1] \to G_2$, auf den eine zu $u$ inzidente Kante $\{u,v\} \in E_1$ durch eine Graph-Zuordndung $s$ abgebildet wird.
Dass diese Randpunkte innerhalb der $\varepsilon$-Platzierung ihrer jeweiligen Knoten liegen, ist ein notwendiges Kriterium dafür, dass der (schwache) Fréchet-Abstand
zwischen der Kante und ihrem Bild unter $s$ nicht größer als $\varepsilon$ ist. Aus der Definition des (schwachen) Fréchet-Abstands (siehe \ref{Definition Fréchet-Abstand}) folgt direkt, dass
$\delta_{(w)F}(e, w) \geq max{\{\|u-w(0)\|, \|v-w(1)\|\}}$.
\\
\\
In diesem Sinne berechnen wir zunächst alle $\varepsilon$-Platzierungen der Knoten in $G_1$ und bezeichnen für einen Knoten $v \in V_1$ mit $P(v)$ die Menge aller $\varepsilon$-Platzierungen von $v$.

\subsubsection{Schritt 1}
Wir initialisieren $\varepsilon$ mit $\varepsilon \geq \vec{\delta}_{(w)F}(G_1, G_2)$. Da $G_1$ ein Baum ist, können wir
$\vec{\delta}_{wF}(G_1, G_2)$ in $O(...)$ und $\vec{\delta}_F(G_1, G_2)$ in $O(...)$ berechnen. Siehe dazu ...
Anschließend berechnen wir die $\varepsilon$-Platzierungen aller Knoten in $G_1$. Jeder Knoten in $G_1$ hat $O(m_2)$
$\varepsilon$-Platzierungen, womit insgesamt $(n_1*m_2)$ $\varepsilon$-Platzierung zu berechnen sind.
Für jeden Knoten lassen sich seine $\varepsilon$-Platzierungen über einen Standard-Algorithmus zur Berechnung von
zusammenhängenden Komponenten in Graphen ermitteln, dessen Laufzeit entsprechend linear in der Größe von $G_2$ ist.
Die Laufzeit - und Speicherkomplexität betragen damit $O(n_1*m_2)$ (siehe [2], Seite 9).

\subsubsection{Schritt 2}
Für jede Kante $\{u, v\} \in E_1$ berechnen wir nun die Erreichbarkeiten zwischen den $\varepsilon$-Platzierungen der Knoten von $\{u,v\}$.
Dabei ist für jede Kombination $p_u, p_v$ von $\varepsilon$-Platzierungen von $u$ und $v$ zu jeweils zu entscheiden, ob sie voneinander (schwach) erreichbar sind.

Um die schwache Erreichbarkeit zu entscheiden, ist die Existenz eines einfachen Weges zwischen zwei Knoten-Platzierungen
$p_u,p_v$, welcher vollständig innerhalb von $T_{\varepsilon}(\{u,v\})$ liegt, bereits ein hinreichendes Kriterium. (Vergleich hierfür [3] Seite 83).

Schränken wir $G_2$ auf $T_{\varepsilon}(\{u,v\})$ ein, so sind alle Knoten-Platzierungen innerhalb derselben zusammenhängenden Komponente von $G_2$
schwach voneinander erreichbar. Entsprechend speichern wir für eine zusammenhängende Komponente von $G_2$ innerhalb $T_{\varepsilon}(e)$ jeweils zwei Listen mit den
der Knoten-Platzierungen $P(u)$ und $P(v)$.

Die schwachen Erreichbarkeiten zwischen allen Knoten-Platzierungen entlang einer Kante in $G_1$ lassen sich somit mit einer Laufzeit von $O(m_2)$ berechnen.
Insgesamt beträgt die Zeit - und Speicherkomplexität zur Berechnung der Erreichbarkeiten aller Kanten in $G_1$ $O(m_1*m_2)$.
\\
\\
Für den Fréchet-Abstand ist die Existenz eines einfachen Weges $w$ innerhalb von $T_{\varepsilon}(\{u,v\})$, welcher $p_u$ und $p_v$ miteinander verbindet
lediglich ein notwendiges Kriterium dafür, dass $p_u$ und $p_v$ voneinander erreichbar sind. Wir müssen für $w$ explizit entscheiden,
ob $\delta_F(\{u,v\}, w) \leq \varepsilon$.
\\
\\
Dabei können wir für zwei Wege $Y,Z$ die Ungleichung $\delta_F(Y,Z) \leq \varepsilon$ in $O(\|Y\|*\|Z\|)$ entscheiden (Siehe [3], Seite 77).
Da $\{u,v\}$ ein Geradenstück ist, gilt $\|\{u,v\}\| = 1$. Damit können wir die Erreichbarkeit zwischen $\{u,v\}$ und $w$ in $O(|w|)$ entscheiden.
Da $w$ per Bedingung ein einfacher Weg ist, besteht $w$ aus maximal $m_2$ Kanten bzw. Teilkanten von $G_2$.
Zur Speicherung der Erreichbarkeiten speichern wir für jede $\varepsilon$-Platztierung von $u$ eine Liste mit $\varepsilon$-Platzierungen von $v$, die
über ein $w$ voneinander erreichbar sind. Um die Liste für ein $p_u \in P(u)$ zu berechnen, starten wir eine Graph-Suche zur Ermittlung eines
Weges in $G_2$ beginnend bei $p_u$. Verlassen wir während der Suche entweder $T_{\varepsilon}(\{u,v\})$ bzw. überschreitet der Fréchet-Abstand $\varepsilon$,
so entfernen wir den Suchzweig aus der Liste der möglichen Abzweigungen in $G_2$.
Für eine Kante in $G_1$ beträgt die Laufzeit der Suche somit $O(m_2^2)$ und pro Kante haben wir bis zu $m_2^2$ Paare voneinander erreichbarer Knoten-Platzierungen.
Für alle Kanten in $G_1$ liegt die Speicher - und Laufzeitkomplexität damit bei $O(m_1*m_2^2)$.

\subsubsection{Existenz einer $\varepsilon$-Platzierung von $G_1$}
Bis zu diesem Schritt haben wir die (schwachen) Erreichbarkeiten zwischen Knoten-Platzierungen paarweise ermittelt.
Nun wollen wir unseren Wortschatz um den Begriff der Knoten-Platzierungen insofern erweitern, dass wir in unserem konkreten Fall die Existenz einer
$\varepsilon$-Platzierung von $G_1$ garantieren können.
\\
Falls nämlich z.B. der Knoten $u$ neben $\{u,v\}$ nun noch eine weitere inzidente Kante $\{u,w\} \in E_1$ hat, so muss neben einem $p_v \in P(v)$ insbesondere auch
ein $p_w \in P(w)$ existieren, so dass die Paare $\{p_u,p_v\}$ und $\{p_v,p_u\}$ (schwach) voneinander erreichbar sind. Die paarweise (schwache) Erreichbarkeit
zwischen Knoten-Platzierungen von einander adjazenten Knoten ist offensichtlich ein notwendiges Kriterium für die Existenz einer $\varepsilon$ von $G_1$.
Dies motiviert die folgende Definition.
\begin{Def}
	Gültige $\varepsilon$-Platzierung (engl. valid placement)
	Eine $\varepsilon$-Platzierung $p_v$ eines Knotens $v$ nennen wir (schwach) gültig, falls für jeden Nachbarn $u$ von $v$
	eine $\varepsilon$-Platzierung $p_u$ existiert, so dass $p_v$ und $p_u$ (schwach) voneinander erreichbar sind.
	Ansonsten nennen wir $p_v$ (schwach) ungültig.
\end{Def}

Für die Konstruktion von $s$ dürfen wir nur gültige $\varepsilon$-Platzierungen der Knoten von $G_1$ betrachten, denn die
Wahl einer ungültigen Knoten-Platzierung schließt definitionsgemäß direkt aus, dass $s$ eine $\varepsilon$-Platzierung von $G_1$ sein kann.
Entsprechend löschen wir unter den in Schritt 1 ermittelten Knoten-Platzierungen alle diejenigen, die (schwach) ungültig sind.
\\
\\
Dabei ist zu beachten, dass nach dem Löschen einer (schwach) ungültigen $\varepsilon$-Platzierung von $v \in P(v)$
jede vor der Löschung noch (schwach) gültige $\varepsilon$-Platzierung eines Nachbarn von $v$ nun (schwach) ungültig geworden sein kann.
Unter Umständen war z.B. die Gültigkeit der Knoten-Platzierung eines Nachbarn von $v$ gerade durch die Erreichbarkeit zu der zuvor gelöschten
$\varepsilon$-Platzierung von $v$ bedingt.
In dem Sinne löschen wir rekursiv solange (schwach) ungültige $\varepsilon$-Platzierungen der Knoten in $G_1$, bis
keine (schwach) ungültigen Knoten-Platzierungen mehr existieren. Als Abbruchbedingung der Rekursion gilt daher, dass die Löschung der letzten bekannten
ungültigen Knoten-Platzierung keine weitere Löschung mehr impliziert.

\subsubsection{Schritt 3} \label{Schritt 3}
In diesem Schritt werden wie oben beschrieben alle Knoten-Platzierungen (rekursiv) bereinigt.
Um für eine Kante $\{u,v\} \in E_1$ zu entscheiden, welche Platzierungen der Knoten $u$ und $v$ voneinander (schwach) erreichbar sind, bedienen wir uns dafür der in
Schritt 2 berechneten Erreichbarkeiten. In Schritt 1 haben wir $O(n_1*m_2)$ Knoten Platzierungen ermittelt.

Für die schwache Erreichbarkeit haben wir in Schritt 2 für eine Kante $\{u,v\} \in E_1$ die zusammenhängende Komponente innerhalb von $T_{\varepsilon}(\{u,v\})$
und ihre enthaltenen Knoten-Platzierungen von $u$ und $v$ gespeichert.
\\
Ist nun eine Knoten-Platzierung $p_v \in P(v)$ ungültig, so muss $p_v$ aus allen entsprechenden Listen von Knoten-Platzierungen gelöscht werden, in
denen $p_v$ enthalten ist. Dies betrifft potentiell alle zu $v$ inzidenten Knoten, daher beträgt die Laufzeit zum Bereinigen von $p_v$ $O(deg(v))$.
Mit der Identität $\sum{v \in V_1}(deg(v)) = 2m_1$ beträgt die Laufzeit zum Bereinigen aller ungültigen Knoten-Platzierungen $O(m_1*m_2)$.
\\
Für die nicht-schwache Erreichbarkeit haben wir für jede Knoten-Platzierung $p_v$ eine Liste mit erreichbaren Knoten-Platzierungen gespeichert.
Sollte $p_v$ ungültig sein, so müssen wir $p_v$ für jeden zu $v$ adjazenten Knoten aus den Listen der Knoten-Platzierungen entfernen, mit denen $p_v$ erreichbar ist.
Für jeden ajdazenten Knoten sind dies $O(m_2)$ Listen, daher beträgt die Laufzeit für die Löschung von $p_v$ $O(deg(v)*m_2)$.
Das Bereinigen aller Platzierungen beträgt entsprechend $O(m_1*m_2^2)$.
\\
\\
Nachdem Schritte 1-3 durchgeführt wurden, liefert uns ein Hilfssatz aus Akitaya et al. (siehe dazu Seite 13. Lemma 6) eine verbindliche Aussage über die Existenz
einer (schwachen) $\varepsilon$-Platzierung von $G_1$:

\begin{Lem} \label {Lemma 1}
	Sei $G_1$ ein Baum und für $v \in V_1$ sei $\tilde{P}(v)$ die Menge der (schwach) gültigen $\varepsilon$-Platzierungenvon $v$ nachdem
	alle (schwach) ungültigen ungültigen $\varepsilon$-Platzierungen der Knoten in $G_1$ rekursiv gelöscht wurden (entsprechend \ref{Schritt 3}.
	Enthält $\tilde{P}(v)$ für jedes $v \in V_1$ mindestens eine (schwach) gültige $\varepsilon$-Platzierung, so hat $G_1$ eine schwache $\varepsilon$-Platzierung.
\end{Lem}

Beweis:
Für den Beweis konstruieren wir eine $\varepsilon$-Platzierung von $G_1$.
Wir wählen einen beliebigen Knoten $v_r \in V_1$ zur Wurzel und betrachten $G_1$ als gerichteten Baum ausgehend von $v_r$.
Wir ordnen $v_r$ einem beliebigen Punkt innerhalb einer seiner (schwachen) $\varepsilon$-Platzierung zu und bearbeiten anschließend iterativ alle
Knoten in $G_1$ ausgehend von der Wurzel.

Dabei wählen wir für einen Knoten $v \in V_1$, dessen Vorgänger bereits einer seiner (schwachen) $\varepsilon$ zugewiesen wurde,
seine $\varepsilon$-Platzierung $p_v \in P(v)$ so, dass $p_v$
und die bereits ausgewählte (bzw. zugeordnete) $\varepsilon$-Platzierung $p_a$ des Vorgängers von $v$ voneinander (schwach) erreichbar sind.
Für die Kante zwischen $v$ und seinem Vorgänger können wir nun eine passende Kanten-Platzierungen wählen.
Denn nach Vorraussetzung ist $p_a$ (schwach) gültig und somit muss ein $p_v$ mit der geforderten Erreichbarkeit zu $p_a$ existieren.
Ferner gilt nach Definition der (schwachen) Erreichbarkeit, dass mindestens ein einfacher Weg in $G_2$ existiert, welcher $p_v$ und $p_a$ verbindet
und dessen Abstand zur der Kante zwischen $v$ und seinem Vorgänger bezüglich des (schwachen) Fréchet-Abstands nicht größer als $\varepsilon$ ist.
Einen solchen Weg wählen wir nun als $\varepsilon$-Platzierung für die Kante zwischen $v$ und seinem Vorgänger.
\\
Da wir $G_1$ als gerichteten Baum betrachten, sind nach endlich vielen Schritten ausgehend von der Wurzel alle Knoten in $G_1$
entsprechend der beschriebenen Vorgaben zugewiesen und ihre $\varepsilon$-Platzierungen
wurden mit den $\varepsilon$-Platzierungen ihrer Nachfahren durch einen einfachen Weg in $G_2$ verbunden.
Da $G_1$ als Baum insbesondere kreisfrei ist, wurde explizit jede Kante in $G_1$ einer $\varepsilon$-Platzierung zugeordnet.
Die durch die Auswahl dieser $\varepsilon$-Platzierungen implizierte Graph-Zuordnung ist damit insgesamt eine $\varepsilon$-Platzierung von $G_1$. $\qed$

\subsubsection{Konstruktion einer optimalen $\varepsilon$-Platzierung von $G_1$}
Über die Konstruktion aus dem Beweis von Lemma \ref{Lemma 1} erhalten wir einen ersten Ansatz zur Konstruktion einer $\varepsilon$-Platzierung gemäß der Einschränkungen und
des Optimalitätskriteriums.

Die unter \ref{Zweite Einschränkung} geforderten Fixpunkte ließen sich z.B. vor der Berechnung der Kanten-Platzierungen entsprechend setzen.
Im Beweis von $\ref{Lemma 1}$ konnten wir zur Konstruktion einer $\varepsilon$-Platzierung von $G_1$ stets eine beliebige $\varepsilon$-Platzierung einer Kante wählen.
\\
\\
Hier können wir die Konstruktion entsprechend anpassen, indem wir jeweils nur eine lokal optimale $\varepsilon$-Platzierung einer Kante (vergleich 4.2.2) in Betracht ziehen.
Konkret gemeint ist damit, dass eine optimale Kanten-Platzierung $p_{\{u,v\}}$ den (schwachen) Fréchet-Abstand minimiert.
Für ein paar Knoten-Platzierungen $p_u, p_v$ sei $$\Delta(p_u,p_v) = \min_{p_{{\{u,v}\}}} \delta_{(w)F}(\{u,v\},p_{\{u,v\}}),$$ wobei sich $p_{{\{u,v}\}}$ über alle
$\varepsilon$-Platzierungen von $\{u,v\}$ erstreckt.

Sofern $\Delta$ für alle gültigen und voneinander erreichbaren Paaren von Knoten-Platzierungen bekannt ist,
reduziert sich die Konstruktion einer optimalen $\varepsilon$-Platzierung auf eine konkrete Auswahl der Knoten-Platzierungen von $G_1$.
Dies ist ähnlich wie im Beweis zu Lemma $\ref{Lemma 1}$, allerdings ist die Auswahl einer Kanten-Platzierung hier nicht willkürlich, sondern bereits optimiert.
Letztlich stellt dann die Frage, unter welcher der möglichen Auswahlen der Knoten-Platzierungen die Summe der Abstände zwischen den Kanten ihren Platzierungen minimal wird.
\\
\\
Ein naiver Ansatz zur Lösung des Problems wäre, alle möglichen Kombinationen der $\varepsilon$-Platzierungen der Knoten von $G_1$ über z.B. einen Backtracking-Algorithmus auszuschöpfen.
Dabei würden wir - beginnend bei der Wurzel - die Knoten und Kanten von $G_1$ wie in der Konstruktion der Graph-Zuordnung in $\ref{Lemma 1}$ (unter Einhaltung der Einschränkungen
und lokal optimaler Kanten-Platzierungen) zuordnen.
An jeder Stelle, wo die Auswahl einer Knoten-Platzierung nicht eindeutig ist, hätte der Backtracking-Algorithmus entsprechende Abzweigungen.
Sobald wir erstmalig $G_1$ vollständig zugeordnet haben, speichern wir die Zuordnung $s$ sowie ihr bisheriges Gewicht $Sum_{(w)F}(s)$, welches sich als Summe der
paarweise gewählten Knoten-Platzierungen und ihren $\Delta$-Gewichten ergibt.
Für einen bereits zugeordneten Teilgraphen könnten wir sein Gewicht unter den bereits zugeordneten Kanten mitführen und backtracken, sobald dieses das bis zu diesem Iterationsschritt
minimalste Gewicht einer vollständigen Zuordnung von $G_1$ überschreitet. Nachdem der Algorithmus terminiert, hätten wir so Min-Sum$_{(w)F}(G_1,G_2)$ berechnet.

Allerdings ist trotz der Einschränkungen and die zu betrachteten Graph-Zuordnungen der Möglichkeitenraum im Allgemeinen zu groß
für einen solchen (brute-force) Ansatz. Jeder Knoten in $G_1$ hat potentiell bis zu $m_2$ $\varepsilon$-Platzierungen und für
zwei benachbarte Knoten in $G_1$ gibt somit bis zu $m_2^2$ Möglichkeiten, die Knoten-Platzierungen zu kombinieren.
Insgesamt wächst die Anzahl der potentiellen Kombinationsmöglichkeiten exponentiell mit der Anzahl der Kanten bzw. der Anzahl der Knoten(**) von $G_1$.
(**) Da $G_1$ ein Baum ist, gilt $m_1 = n_1-1$.
\\
\\
Wir machen folgende Beobachtung:
Da der oben beschriebene Backtracking-Algorithmus keine polynomielle Laufzeit garantiert, wählen wir einen Ansatz, welcher das kombinatorische
Problem einer Auswahl von $\varepsilon$-Platzierungen greedy löst.
\\
Anders: Wir finden heraus, dass sich die Konstruktion über eine Änderung der Reihenfolge greedy lösen lässt.
Diesen Ansatz könnte man über eine Beobachtung motivieren.
\\
Hier könnten wir den Algorithmus bereits beschreiben oder den Ansatz über einen Hilfssatz zeigen.
\begin{Lem}
	Sei $G_1$ ein Baum und $\varepsilon \geq \vec{\delta}_{(w)F}(G_1, G_2)$. Für jedes $v \in V_1$ sei die Menge seiner gültigen
	$\varepsilon$-Platzierungen $P(v)$ sei jede seiner  (schwach) gültig.
\end{Lem}

Dabei interpretieren wir $G_1$ wie in $\ref{Lemma 1}$ als gerichteten
Wurzel-Baum (Out-Tree). Ferner nehmen wir an, dass alle $\varepsilon$-Platzierungen der Knoten von $G_1$ gültig sind und das alle $\varepsilon$-Platzierungen der Kanten
in $G_1$ fixierte Randpunkte haben und lokal optimal gewählt sind.

\subsection{Beschreibung des Algorithmus}
Den in \ref{Grundidee des Algorithmus} beschriebenen Ansatz wollen wir nun formalisieren. Dabei werden wesentliche Berechnungsschritte des Entscheidungs-Algorithmus aus [2]
um die zur Berechnung des Min-Sum-Graph-Abstands$_{(w)F}$ notwendigen Anpassungen aus [1] erweitert.
Der Algorithmus unterteilt sich letztlich in die folgenden 4 Einzelschritte:

\begin{enumerate}
		\item[1)] Schritt 1: Berechnung der $\varepsilon$-Platzierungen der Knoten von $G_1$
		\item[2)] Schritt 2: Berechnung der (schwachen) Erreichbarkeiten
		\item[3)] Schritt 3: Löschung aller ungültigen $\varepsilon$-Platzierungen der Knoten von $G_1$
		\item[4)] Schritt 4: Konstruktion einer Min-Sum$_{(w)F}$ Zuordnung
\end{enumerate}

\subsubsection{Schritt 4}
Hier beschreiben wir die Konstruktion im Detail und beweisen ihre Korrektheit.
\\
\\
Hier schreiben wir erstmalig etwas zu den Fixpunkten
Abschließend schreiben wir das Lemma auf.

\subsection{Ein Beispiel}
Anhand eines kleinen Beispiels wollen wir den eben beschriebenen Algorithmus nun demonstrieren.

\section{Komplexität}

\subsubsection{NP-Schwerheit von Min-Sum-Graph-Abstand$_{(w)F}$}

Abschließend wollen wir zeigen, dass die Berechnung des Min-Sum-Graph-Abstands$_{(w)F}$ ohne Bedingungen an die Graphen $G_1$ und $G_2$ im Allgemeinen
NP-Schwer ist. Ein für die Konstruktion des Beweises ähnliches Setting finden wir in [2] (Seite), wo die NP-Schwerheit des Graph-Abstands
über eine Reduktion von \textit{Binary Constraint Satifaction Problem (CSP)} gezeigt wird.
In unserem Beweis der NP-Schwerheit für den Min-Sum-Graph-Abstand orientieren wir uns im Wesentlichen an dieser Konstruktion.

\begin{Def}
	Binary Constraint Satisfaction Problem
	Ein \textit{Binary Constraint Satisfaction Problem} beschreibt folgendes Entscheidungsproblem:
	\\
	Gegeben eine Instanz $\langle X,D,C \rangle$ bestehend aus
	$$ \text{einer Menge von \textit{Variablen }}X = \{x_1, x_2, ..., x_n\},$$
	$$ \text{einer Menge von \textit{Domänen }}D = \{D_1, D_2, ..., D_n\} $$
	$$ \text{und einer Menge von \textit{Bedingungen }}C = \{C_1, C_2, ..., C_k\}. $$
	Für jede Variable $ x_i \in X$ beschreibt die Domäne $ D_i \in D$ die Menge ihrer möglichen Wertzuweisungen.
	Eine Bedingung $C_{i,j} \in C$ spezifiert für je zwei unterschiedliche Variablen $x_i, x_j \in X$ eine Relation $R_{C_{i,j}}$ $\subseteq D_v \times D_w$.
	Wir nennen ein Wertepaar $(d_i, d_j) \in D_i \times D_j$ für eine vorhandene Bedingung $C_{i,j}$ an die Variablen $x_i,x_j$ \textit{zulässig}, wenn $(d_i,d_j) \in R_{C_{i,j}}$,
	ansonsten \textit{verletzt} das Wertepaar $(d_i, d_j)$ die Bedingung $C_{i,j}$ und wir nennen $(d_i,d_j)$ \textit{unzulässig}.
	\\
	Die Fragestellung ist nun, ob alle Variablen einem Wert aus ihrer Domäne zugewiesen werden können, so dass alle mit Bedingungen versehenden Wertepaare zulässig sind.
	In diesem Falle nennen wir $\langle X,D,C \rangle$ \textit{lösbar}.
	\\
	\\
	TODO: Referenz, dass CSP NP-Schwer ist
\end{Def}

\begin{Satz}
	Theorem: Seien $G_1$ und $G_2$ geometrische Graphen und sei $\varepsilon \geq 0$.
	Das Entscheidungsproblem $$ \textit{Min-Sum-Graph-Abstand}_{(w)F}(G_1, G_2) \leq  \varepsilon $$ ist NP-Schwer.
\end{Satz}

Sei $\langle X,D,C \rangle$ ein beliebiges Binary Constraint Satisfaction Problem.
\\
\\
Wir repräsentieren jede Variable $x_i \in X$ durch einen Knoten $v_i \in V_1$ in $G_1$ und für jede Bedingung $C_{i,j}$ an die Variablen $x_i, x_j$
hat $G_1$ die Kante $\{v_i, v_j\}$. Wir setzen $\varepsilon = |E_1|$.
\\
Wir betten $G_1$ so in die euklidische Ebene ein, dass sich für je zwei Knoten $v_i,v_j$ ihre $\varepsilon$-Bälle nicht berühren und
sich der $\varepsilon$-Schlauch einer Kante $e \in E_1$ genau dann mit dem $\varepsilon$-Ball eines Knotens $v \in V_1$ überlappt, wenn $e$ und $v$ inzident sind.
\\
\\
Eine solche Einbettung für $G_1$ lässt sich z.B. realisieren, indem wir die Knoten in $G_1$ (gleichmäßig) auf einem Kreis mit entsprechend großem Radius verteilen.
\\
Jeder Wert $d_{i,a} \in D_i$ wird durch einen Knoten $u_{i,a}$ in $G_2$ repräsentiert und wir betten $_{i,a}$ beliebig innerhalb des 1-Balles $B_1(v_i)$ von $v_i$ ein.
Für jedes Wertepaar $d_{i,a} \in D_i, d_{j,b} \in D_j$ enthält $G_2$ genau dann die Kante $\{u_{i,a},u_{j,b}\}$, wenn die Wertekombination $(d_{i,a},d_{j,b})$
zulässig ist.
\\
Alle einzubettenen Kanten von $G_1$ und $G_2$ ergeben sich implizit (Geradenstücke) zwischen den entsprechenden Knoten-Einbettungen.
Insgesamt behaupten wir, dass sich die oben beschriebene Einbettung von $G_1$ und $G_2$ in Polynomialzeit realisieren lässt.
\\
\\
TODO: Figure XX zeigt die Konstruktion anhand eines kleinen Beispiels
\\
\\
Wir wollen nun zeigen, dass
$$ \min_{s: G_1 \to G_2} \sum_{e \in E_1} \delta_{(w)F}(e, s(e)) \leq \varepsilon \iff \langle X,D,C \rangle \text{ ist lösbar}$$
"$\Leftarrow$":
\\
Sei $\langle X,D,C \rangle$ lösbar.
\\
Dann existiert eine Wertezuweisung $(\tilde{d_1},\tilde{d_2},...,\tilde{d_n}) \in {D_1 \times D_2 \times ... \times D_n},$ welche keine der Bedingungen verletzt.
\\
\\
Wir definieren eine Abbildung $\tilde{s}:G_1 \to G_2$ mit $\tilde{s}(v_i) = u_{\tilde{d_i}}$.
Nach Konstruktion von $G_2$ existiert für jedes $\tilde{d_i}$ ein entsprechendes $u_{\tilde{d_i}} \in G_2$.
\\
Ferner existiert für eine beliebige Kante $\{v_i, v_j\} \in E_1$ die Kante $\{u_{\tilde{d_i}}, u_{\tilde{d_j}}\} \in E_2$ und diese ist
und da das mit $\{u_{\tilde{d_i}}, u_{\tilde{d_j}}\}$ assoziierte Wertepaar $(\tilde{d_i},\tilde{d_j})$ per Vorraussetzung zulässig ist,
liegt $\{u_{\tilde{d_i}}, u_{\tilde{d_j}}\}$ innerhalb von $T_1(\{v_i, v_j\}$ und die Randpunkte liegen respektive innerhalb von $B_1(v_i)$ und $B_1(v_j)$.
\\
\\
Wir erweitern $\tilde{s}$ insofern, dass $\{v_i, v_j\}$ durch $s$ so auf $\{u_{\tilde{d_i}}, u_{\tilde{d_j}}\}$ abgebildet wird,
dass $\delta_{(w)F}(\{v_i, v_j\}, s(\{v_i, v_j\}) \leq 1$. Bilden wir jede Kante von $G_1$ auf diese Weise auf $G_2$ ab, so ist $\tilde{s}$ eine Graph-Zuordnung
und $\delta_{(w)F}(e, \tilde{s}) \leq 1$ für jedes $e \in E_1$.
\\
\\
Damit ist $$\sum_{e \in E_1} \delta_{(w)F}(e, \tilde{s}(e)) \leq \varepsilon \text{ (i)} $$
\\
Über die Identität $$\min_{s: G_1 \to G_2} \sum_{e \in E_1} \delta_{(w)F}(e, s(e)) \leq \sum_{e \in E_1} \delta_{(w)F}(e,\tilde{s}(e)) $$
und (i) erhalten wir
$$\min_{s: G_1 \to G_2} \sum_{e \in E_1} \delta_{(w)F}(e, s(e)) \leq \varepsilon .$$
\\
\\
"$\Rightarrow$":
\\
Sei $$\min_{s: G_1 \to G_2} \sum_{e \in E_1} \delta_{(w)F}(e, s(e)) \leq {\varepsilon}.$$
\\
\\
So ist $s(e)$ für alle $e \in E_1$ eine (schwache) $\varepsilon$-Platzierung von $e$,
da ansonsten $\delta_{(w)F}(\tilde{e}, s(\tilde{e})) > \varepsilon$ für ein $\tilde{e} \in E_1$ und damit insbesondere $\sum_{{e}\in E_1} \delta_{(w)F}(e, s(e)) > \varepsilon$.
(*)
\\
\\
Zusätzlich wollen wir argumentieren, dass $s(e)$ für alle $e \in E_1$ eine (schwache) 1-Platzierung von $e$ ist.

Wir haben bereits argumentiert, dass jedes Bild einer Kante von $G_1$ unter $s$ eine $\varepsilon$-Platzierung sein muss.
Damit liegen die Randpunkte von $s(e)$ insbesondere innerhalb der $\varepsilon$-Platzierungen der zu $e$ inzidenten Knoten.
Wenn $s(e)$ eine $\varepsilon$-Platzierung ist von $e$, so muss $s(e)$ insbesondere auch eine 1-Platzierung von $e$ sein.
\\
Dies ergibt sich aus der Vorraussetzung, dass $s$ eine Min-Sum$_{(w)F}$ Zuordnung ist und dass aufgrund der Konstruktion
von $G_1$ und $G_2$ jede $\varepsilon$-Platzierung von $e$ sich grundsätzlich auch als 1-Platzierung von $e$ realisieren lässt.
\\
Denn die Randpunkte von $s(e)$ lassen sich innerhalb der 1-Bälle versetzten und da alle Kanten von $G_2$ innerhalb von $T_{\varepsilon}(e)$
per Konstruktion auch innerhalb von $T_1(e)$ verlaufen, existiert in $G_2$ eine 1-Platzierung von $e$.
\\
\\
Damit ist für eine Kante $\{v_i, v_j\} \in E_1$ ihr Bild $s(\{(v_i, v_j)\})$ stets eine 1-Platzierung von $\{v_i, v_j\}$. (**)
\\
\\
Im Falle des Fréchet-Abstands gilt damit, dass wir durch eine $\text{Min-Sum}_F$ Zuordnung $s: G_1 \to G_2$ das Bild $s(\{v_i, v_j\})$ einer Kante
eindeutig mit einer Kante $\{u_{i,a}, u_{j,b}\} \in E_2$ identifizieren können. Gemeint ist damit genau die Kante, welche
die entsprechenden $1$-Platzierungen der Knoten $v_i$ und $v_j$ verbindet und in $T_1(\{v_i, v_j\})$ liegt. Die Eindeutigkeit dieser Kante folgt direkt aus der
Konstruktion von $G_2$.
\\
Durch (**) ist gewährleistet, dass für die Kante $\{u_{i,a}, u_{j,b}\} \in E_2$, ihren assoziierten Wertezuweisungen $(d_{i,a}, d_{j,b}) \in D_i \times D_j$
und die Bedingung $C_{i,j} \in C$ an die Variablen $x_i, x_j$ gilt:
$$(d_{i,a},d_{j,b}) \in R_{C_{i,j}}$$
Damit ist im Falle des Fréchet-Abstands $\langle X,D,C \rangle$ erfüllbar.
\\
\\
Für den schwachen Fréchet-Abstand gilt das Argument, dass wir das Bild einer Kante in $E_1$ unter $s$ eindeutig mit einer Kante in $E_2$ identifizieren können, allgemein nicht.

Dies liegt daran, dass grundsätzlich jedes beliebige Knotenpaar $u_{i,a}, u_{i,b} \in V_2$ mit $u_{i,a} \in B_1(v_i)$, $u_{i,b} \in B_1(v_j)$ eine Kante in $G_2$ haben kann.
\\
So existieren Wege in $G_2$ bestehendend aus mindestens 3 Kanten innerhalb von $T_1(\{v_i, v_j\}$, deren Startpunkt in $B_1(v_i)$ und deren Endpunkt in $B_1(v_j)$ liegt.
Ein solcher Weg $W$ ist dabei möglicherweise - da er vollständig innerhalb von $T_1(\{v_i, v_j\}$ liegt - eine schwache 1-Platzierung der Kante $\{v_i, v_j\}$, also
$$ \delta_{wF}(\{v_i, v_j\}, W) \leq 1.$$
\\
\\
TODO: Skizze
\\
\\
Um dies zu umgehen, fügen wir mittig auf jeder Kante $\{v_i, v_j\} \in E_1$ einen zusätzlichen Knoten $v_{ij}$ in $G_1$ ein,
so dass $B_1(v_{ij}) \cap B_1(v_i) = B_1(v_{ij}) \cap B_1(v_j) = \emptyset$.
\\
\\
Hinweis: Hierfür können wir initial auch $\varepsilon = \max \{m_1, 4\}$ fordern.
\\
\\
Die Kante $\{v_i, v_j\}$ zerfällt dabei in die zwei Kanten $\{v_i, v_{ij}\}$, $\{v_{ij},v_j\}$ und
jede 1-Platzierung von $\{v_i, v_j\}$ zerfällt dabei zwei 1-Platzierungen der Kanten $\{v_i, v_{ij}\}$ und $\{v_{ij}, v_j$\}.
\\
\\
TODO: Skizze
\\
\\
TODO: Ergänze Erklärung, warum das Problem jetzt gelöst ist.
\\
\\
Wir bezeichnen mit $\tilde{G_1}=(\tilde{V_1}, \tilde{E_1})$ den Graphen, den wir durch die oben beschriebene Transformation erhalten.
\\
Gilt nun
$$ \min_{s: \tilde{G_1} \to G_2} \sum_{e \in E_1} \delta_{wF}(e, s(e)) \leq 2m_1, $$
so erhalten wir für jede Kante $\{v_i, v_j\} \in E_1$ und ihre Zerlegung $\{v_i, v_{ij}\},\{v_{ij}, v_j\} \in \tilde{E_1}$
über $s(\{v_i, v_{ij}\},\{v_{ij}, v_j\})$ die eindeutige, assoziierte Kante $\{u_{i,a}, u_{i,b}\} \in E_2$.
Die durch $\{u_{i,a}, u_{j,b}\} \in E_2$ repräsentierte Wertezuweisung $(d_{i,a}, d_{j,b})$ verletzt dabei nicht die Bedingung an die
durch $v_i$ und $v_j$ repräsentierten Variablen $x_i, x_j$ und ist damit gültig.
\\
\\
Damit ist $\langle X,D,C \rangle$ lösbar. \qed
\\
\\
\section{Ausblick}
In dieser Arbeit haben wir uns, motiviert durch das Ziel optimale Zuordnungen auf geometrischen Graphen zu untersuchen, mit dem
Min-Sum-Graph-Abstand$_{(w)F}$ beschäftigt. Dabei haben wir die Frage der NP-Schwerheit des Min-Sum-Graph-Abstands$_{(w)F}$ für allgemeine
Graphen geklärt und - falls $G_1$ ein Baum ist und wir die Menge der Graph-Zuordnungen (wie in \ref{Einschränkungen} beschrieben) einschränken -
einen Polynomialzeit-Algorithmus zur Berechnung von Min-Sum-Graph-Abstand$_{(w)F}(G_1, G_2)$ beschrieben.
\\
Dabei verbleiben noch einige offene Fragen bezüglich der Komplexität.
Buchin et al. gehen des Weiteren davon aus, dass sich der Min-Sum-Graph-Abstand$_{(w)F}$ auch auf planaren Graphen und
auf Bäumen ohne die notwendige Einschränkung (aus \ref{Zweite Einschränkung}) ebenfalls NP-Schwer ist.
Formale Beweise dieser Annahmen stehen dabei noch aus.
\\
\\
Allgemein beschreibt der Min-Sum-Graph-Abstand$_{dist}$ eine ganze Klasse von Abstandsmaßen auf Graphen, deren Mitglieder sich durch die Wahl
eines geeigneten Abstandsmaßes für die Gewichtung der Kanten-Zuordnungen unterscheiden.
Grundsätzlich sind die Eigenschaften des Min-Sum-Graph-Abstands unter anderen Abstandsmaßen zu betrachten. Hier bietet sich z.B. der
diskrete Fréchet-Abstand als geeigneter Kandidat an. Wir behaupten, dass wir aufgrund seiner inherenten Eigenschaft die in $\ref{Zweite Einschränkung}$
geforderten Fixpunkte zur Einhaltung der polynomiellen Laufzeit nicht benötigen.
\\
Wir haben bei unserer Betrachtung des Min-Sum${(w)f}$ Abstands bisher nur ein Mitglied dieser oben beschriebenen Klasse kennengelernt und
damit einen kleinen Einblick in den Raum der offenen Möglichkeiten gewagt.
\newpage\null\thispagestyle{empty}\newpage
%   Eventuell eine Leerseite vor der Literaturangabe, falls diese sonst auf einer Rückseite angegeben würde. (Seitennummer muss ungerade sein!)

\begin{thebibliography}{4}
	\bibitem{Buchin}
		Maike Buchin, Bernhard Kilgus. Distance Measures for Embedded Graphs - Optimal Graph Mappings.
		\textit{European Workshop on Computational Gemoetry,} 2020.

	\bibitem{Akitaya}
		Hugo Akitaya, Maike Buchin, Bernhard Kilgus, Stef Sijben, Carola Wenk. Distance measures for embedded graphs
		\textit{Computational Geometry,} Volume 95, 2021.

	\bibitem{Alt}
		Hemlut Alt, Michael Godau. Computing the Fréchet distance between two polygonal curves.
		\textit{Int. Journal of Computational Geometry and Applications,} 5:75-91, 1995.

	\bibitem{Rote}
		Günter Rote, Lexicographic Fréchet distance.
		\textit{European Workshop on Computational Gemoetry,} 2014.

\end{thebibliography}

% Symbolverzeichnis definieren
%\nomenclature{$c$}{Speed of light in a vacuum inertial frame}
%\nomenclature{$h$}{Planck constant}

%\printnomenclature
\end{document}
