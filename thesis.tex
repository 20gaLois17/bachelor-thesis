% Notwendige Start-Codezeile
\documentclass[a4paper, 12pt, twoside]{article}

% Präambel

% Pakete
\usepackage[left=3cm,right=2.5cm,bottom=3.5cm,top=2.5cm]{geometry} % Ok die Seitenrändereinstellungen passen so.
\usepackage{amsfonts}
\usepackage[utf8]{inputenc} % Kodierung
\usepackage[ngerman]{babel} % Sprache
\usepackage{amssymb}        % Mathematische Symbole wie z.b ganze Zahlen, reelle Zahlen etc.
\usepackage{amsmath}        % Um diverse mathematische Symbole nutzen zu können.
\usepackage{amsthm}         % Um Definitionen, Theoreme, Bemerkungen, Beispiele machen zu können.
\usepackage{mathtools}      % Dieses Paket liefert nützliche Werkzeuge, z.B "defined as equal" - sign uvm.
\usepackage{enumitem}       % Um individuelle Listen zu bauen.
\usepackage{xcolor}         % Um farbigen Text machen zu können. Diesen kann ich nutzen um wichtige persönliche Notizen hervorzuheben.
\usepackage{fancyhdr}       % Um saubere Kopf und Fußzeilen sowie Seitenzahlen zu erzeugen.
\usepackage{setspace}       % Damit kann ich Leerzeilen im Dokument einfügen.
\usepackage{graphicx}       % Einbinden von externen Bildern
% insbesondere beim Inhaltsverzeichnis nötig, da dieses sonst zu
% gequetscht aussieht.


% Symbolverzeichnis
%\usepackage{nomencl}
%\makenomenclature
%\renewcommand{\nomname}{Symbolverzeichnis}


% Globale Festlegungen
\setlength{\topsep}{4ex plus0.5ex minus0.5ex} % Festlegung: Größe der Absätze nach Definition, Theorem, etc.
\setlength\parindent{0pt} % Festlegung: Kein Einschub nach rechts.
\setstretch{1.2} % Festlegung: Zeilenabstand ist 1.2 statt 1.
\raggedbottom % twoside sorgt dafür, dass der Platz, der durch \\ und einer Leerzeile erzeugt wird sehr groß ist und nicht nur eine Leerzeile, was ich nämlich erzielen möchte.
              % Dieses Kommando sorgt dafür, dass dies wiederhergestellt wird und twoside trotzdem wirkt.

% Vorlagen
% 1) [before = \leavevmode\vspace{-\baselineskip}]
% // Um bei Theoremen mit Listen zu starten.

% Eigene Befehle

\newcommand\logeq{\mathrel{\vcentcolon\Longleftrightarrow}} % Dieser Befehl realisiert mittels "\logeq" ein "defnierendes Äquivalenzzeichen".
\newcommand{\ts}{\thinspace} % Damit ich nicht so viel schreiben muss, wenn ich kleine Leerzeilen hinzufügen möchte. Was ich oft tue bei Mengendefinitionen, da mir der Platz dort zu klein ist.

% Eigene Theoremstyles
% Format 1
\newtheoremstyle{Format1}
{\topsep}   % ABOVESPACE
{\topsep}   % BELOWSPACE
{\normalfont}  % BODYFONT
{0pt}       % INDENT (empty value is the same as 0pt)  % Das rückt den Kopf nach rechts ein
{\bfseries} % HEADFONT
{\newline}  % HEADPUNCT
{5pt plus 1pt minus 1pt} % HEADSPACE
{}          % CUSTOM-HEAD-SPEC


% Eigene Theorem-Umgebungen
\theoremstyle{Format1} % Alle Theoremumgebungen hierunter folgen den Spezifikationen vom "Format 1" - Theoremstil
\newtheorem{Def}{Definition}[section]       % Definition
\newtheorem*{Definition}{Definition}        % Definition (unnummeriert)
\newtheorem{Bsp}[Def]{Beispiel}             % Beispiel
\newtheorem{Bem}[Def]{Bemerkung}            % Bemerkung
\newtheorem{Satz}[Def]{Satz}                % Satz
\newtheorem*{Bez}{Bezeichnung}              % Bezeichnung (unnummeriert)
\newtheorem{Folg}[Def]{Folgerung}           % Folgerung
\newtheorem*{Folgerung}{Folgerung}          % Folgerung (unnummeriert)
\newtheorem{Lem}[Def]{Lemma}                % Lemma
\newtheorem*{Grundmodell}{Grundmodell}
\newtheorem*{Aussage}{Aussage}
\newtheorem*{Herleitung}{Herleitung}

% Code aus Stackexchange, welcher bei Nutzen von Norm oder Betrag automatisch passende Betragsgrößen generiert, also die Beträge dem Ausdruck entsprechend vergrößert:

\DeclarePairedDelimiter\abs{\lvert}{\rvert}%
\DeclarePairedDelimiter\norm{\lVert}{\rVert}%

% Swap the definition of \abs* and \norm*, so that \abs
% and \norm resizes the size of the brackets, and the
% starred version does not.
\makeatletter
\let\oldabs\abs
\def\abs{\@ifstar{\oldabs}{\oldabs*}}
%
\let\oldnorm\norm
\def\norm{\@ifstar{\oldnorm}{\oldnorm*}}
\makeatother

% Code aus Stackexchange: Sorgt dafür, dass die Abstände zwischen Zeilen in Allign Umgebungen etwas größer sind, also normal. Die Standardabstände sind mir zu gering.

\addtolength{\jot}{0.5em}


% Beginn des Dokumentes
\begin{document}

\newgeometry{} % Damit die Titelseite nicht von den Einstellungen des geometry packages beeinflusst wird.
% Titelblatt der Arbeit
\begin{titlepage}
	\begin{center}
		\vspace*{1cm}

		\Huge
		\textbf{Bachelorarbeit}

		\vspace{0.5cm}
		\LARGE
		Optimale Zuordnungen auf Geometrischen Graphen

		\vspace{1.5cm}
		\large Sebastian Koletzko\\

		\vspace{1cm}
		Datum: 17.12.23

		\vfill

		\vspace{5cm}

		%\includegraphics[width=0.4\textwidth]{RuhrU}

		\large
		Fakultät für Mathematik
		\\
		Ruhr-Universität Bochum
		\\
		\vspace{0.5cm}
		Prof. Dr. Maike Buchin
		\\
		PD Dr. Daniela Kacso

	\end{center}
\end{titlepage}

\restoregeometry % Damit die Titelseite nicht vom geoemtry-package beeinflusst wird - Endcodezeile.

\newpage\null\thispagestyle{empty}\newpage % Nach der Titelseite kommt immer eine leere Seite, denn die nachfolgende Seite, ist die Rückseite der Titelseite. Und ich möchte nicht auf der Rückseite direkt anfangen.

\thispagestyle{empty} % Damit keine Seitenzahl auf der Erklärung-Seite auftaucht.

\textbf{Eigenständigkeitserklärung:}
\\
Hiermit erkläre ich, dass ich die heute eingereichte Bachelorarbeit selbstständig verfasst und keine anderen als die angegebenen Quellen und Hilfsmittel benutzt sowie Zitate kenntlich gemacht habe.
Bei der vorliegenden Bachelorarbeit handelt es sich um in Wort und Bild völlig übereinstimmende Exemplare. Ich erkläre weiterhin, dass die vorliegende Arbeit noch nicht im Rahmen eines anderen Prüfungsverfahrens eingereicht wurde.
\\
\\
Witten, den 17.12.23
\\
\\
Sebastian Koletzko

\newpage
abstract:
In dieser Arbeit werden wir ein in [1] vorgestelltes Abstandsmaß auf geometrischen Graphen - den sogenannten \textit{min-sum graph distance} - untersuchen.
Dafür wird einleitend das notwendige mathematische Grundgerüst aus [2] formuliert und - darauf aufbauend - ein in Algorithmus
dargestellt, welcher den \textit{min-sum graph distance} zweier Bäume unter einer Einschränkung in der der Abbildung in Polynomialzeit berechnet.
Darüber hinaus zeigen wir die NP-Schwerheit der \textit{min-sum graph distance} für Graphen ohne die besagte Einschränkung in der Abbildung.


% Inhaltsverzeichnis. (Dieses wird nach Sektionen geordnet)
\newpage
\tableofcontents
\newpage\null\thispagestyle{empty}\newpage % Eine Leerseite nach dem Inhaltsverzeichnis.
\section{Einleitung}

Die geometrische Graphentheorie beschäftigt sich ...

Wir betrachten im Allgemeinen endliche, ungerichtete Graphen, welche in den euklidischen Raum eingebettet werden.

Dadurch werden Knoten zu Punkten im $ \mathbb{R}^n $ und Kanten allg. zu Strecken (mit ihren jeweiligen Knoten als Start - und Endpunkt)
Dabei wird die Knotenmenge des Graphen als (endliche) Teilmenge des $ \mathbb{R}^n $ realisiert.

Geometrische Graphen sind ... (das könnte man vllt. später erläutern)
Daher eignen sich geometrische Graphen insbesonders um, um Netzwerke (z.B. von Straßen, Schienen) zu beschreiben.
\\
Für eingebettete Graphen existieren viele bekannte Anwendungsfälle (wie z.B. Straßennetzwerke) im Falle einer Einbettung in die Ebene.
Eine zentrale Fragestellung dabei ist, wie man Abstandsmaße auf geometrischen Graphen definiert und unter welchen Bedingungen sich dieses Maß effizient berechnen lässt.
\\
Um die Ählichkeit zweier eingebetter Graphen zu beschreiben, haben Akitaya et al. \cite{Akitaya} neue Abstandsmaße eingeführt und ihre Eigenschaften untersucht.
\\
Besonderheit bei diesen Abstandmaßen ist, dass sie auf einer stetigen Zuordnung zwischen den zu vergleichenden Graphen basieren und somit auch die Topologie der Graphen berücksichtigen.
Das resultierende Maß beschreibt eine sogenannte "bottleneck-distance".
\\
Darauf aufbauend haben \cite{Buchin} Buchin et al. ein weiteres Abstandsmaß definiert, den sogenannten "min-sum graph distance".
Vor dem Hintergrund, dass die "directed (weak) graph distance" eine bottle-neck distance beschreibt, wird bei der min-sum graph distance das Ziel verfolgt, lokale Abstände zu berücksichtigen.
[Ein kleiner Text über den Stand der Wissenschaft]

In dieser Arbeit werden wir uns mit der min-sum graph distance beschäftigen.

- worum geht es bei Abstandsmaßen auf geometrischen Graphen?
\newpage

TODO: Allgemein sollten wir eine Einbettung beschreiben.

\begin{Def}
	Eingebetter Graph
\end{Def}

\begin{Def}
	Geometrischer Graph
	Unter einem geometrischen Graphen verstehen wir einen abstrakten Graphen $G=(V, E)$ zusammen mit einer Einbettung $ \omega $
    	\begin{enumerate}
		\item[1)] Jeder Knoten $v \in V $ auf einen eindeutigen Punkt in $ \mathbb{R}^n $ repräsentiert wird.
		\item[2)] Jede Kante ${u,v} \in E$ durch das eindeutige Geradenstück der Punkte im $ \mathbb{R}^n. $ repräsentiert wird.
    	\end{enumerate}
	Wenn wir im weiteren Verlauf von einem geometrischen Graphen $G$ sprechen, so unterscheiden wir grundsätzlich nicht zwischen seiner Konnektivität -
	welche durch die abstrakte Beschreibung des Graphen gegeben ist - und seiner konkreten Repräsentation als Teilmenge des euklidischen Raums.
\end{Def}

\begin{Def}
	Einfacher Weg in einem Geometrischen Graphen
	Ein \textit{einfacher Weg} in einem geometrischen Graphen $G$ ist eine stetige Abbildung $ \phi: [0, 1] \rightarrow G$, welche keinen Knoten
	von $G$ mehr als einmal besucht.
\end{Def}

Seien $ G_1=(V_1, E_1) $ und $ G_2=(V_2, E_2) $ gradlinig eingebettete Graphen.
\begin{Def}
	Graph-Zuordnung (engl. graph mapping)
	Wir nennen eine Abbildung $s: G_1 \to G_2 $ eine \textit{Graph-Zuordnung}, falls
    	\begin{enumerate}
		\item[1)] s jeden Knoten $ v \in V_1 $ auf einen Punkt einer Kante von $ G_2 $ abbildet und
		\item[2)] s alle Kanten $ {u,v} \in E_1 $ auf einen einfachen Weg in $G_2$ mit dem Startpunkt $s(u)$ und dem Endpunkt $s(v)$ abbildet.
    	\end{enumerate}

	Eine Graph-Zuordnung s definiert somit stetige Abbildung von $ G_1 $ nach $ G_2 $.\\
	Bemerkung: Eine Graph-Zuordnung ist im Allgemeinen weder injektiv noch surjektiv.
	$ G_1 $ und $ G_2 $ müssen daher insbesondere nicht homeomorph sein.

\end{Def}

TODO: Hier wäre ein Beispiel angebracht

\begin{Def}
	Fréchet-Abstand (engl. fréchet distance)
	Für zwei Kurven $ f, g: [0,1] \to \mathbb{R}^n $ definieren wir den \textit{Fréchet-Abstand} von $f$ und $g$ als
	$$ \delta_F(f,g) =  \inf_{\sigma:[0,1] \to [0,1]} \; \max_{t \in [0,1]} \lVert f(t)-(g(\sigma(t)) \rVert, $$
	wobei sich $\sigma $ über alle orientierungserhaltenen Homeomorphismen erstreckt.
	\\
	\\
	Wir definieren den \textit{schwachen Fréchet-Abstand} von $f$ und $g$ als
	$$\delta_{wF}(f,g) =\inf_{\alpha , \beta :[0,1] \to [0,1]} \; \max_{t \in [0,1]} \lVert f(\alpha(t))-(g(\sigma(t)) \rVert,$$
	wobei sich $\alpha$ und $\beta$ über alle stetigen Abbildungen erstrecken, welche die Endpunkte fixieren, also $\alpha(0) = \beta(0) = 0$
	und $\alpha(1) = \beta(1) = 1$.
	\\
	\\
	Die Notation $ \delta_{(w)F} $ benutzen wir zugunsten der Kompaktheit nachfolgend, wenn wir im Kontext den Fréchet-Abstand sowie den schwachen Fréchet-Abstand zugleich -
	wir schreiben in diesem Fall auch \textit{(schwacher) Fréchet-Abstand} - adressieren wollen.
	\\
	\\
	Zur Anschauung stelle man sich $f$ und $g$ als Kurven im $\mathbb{R}^2$ vor. Der euklidische Abstand zwischen zwischen $f$ und $g$ zum Zeitpunkt $t_0 \in [0,1]$ entspricht genau der Länge der Strecke
	mit den Endpunkten $f(t_0)$ und $g(t_0)$. Bei dem Fréchet-Abstand dürfen wir eine Kurve reparametrisieren, um den maximal angenommenen euklidischen Abstand zwischen $f$ und $g$ zu minimieren.
	Wir werden so einen maximal angenommen Abstand sinngemäß auch als \textit{Flaschenhals-Abstand} bezeichnen.\\

	In der Literator TODO: REFERENZ findet man hierzu oft das Beispiel eines Mannes, der seinen Hund an einer Leine spazieren führt. (Siehe Skizze)
	Beide starten jeweils am Punkt $f(0)$ bzw. $g(0)$ und enden jeweils am Punkt $f(1)$ bzw. $g(1)$.

	Für den Fréchet-Abstand darf einer von beiden auf seinem festgelegten Weg - welcher der Spur der Kurve entspricht - zu jedem Zeitpunkt beliebig beschleunigen oder sogar stehenbleiben, aber nicht die Richtung wechseln.
	Dies mit dem Ziel, die dabei maximal angenommene Länge der Leine möglichst zu minimieren.
	Für den schwachen Fréchet-Abstand dürfen der Mann sowie sein Hund auf ihrem Weg beliebig beschleunigen und auch die Richtung ändern.
	TODO: Beispiel Illustration anfügen
	\\
	Da der Fréchet-Abstand symmetrisch ist !TODO: REFERENZ oder beweisen!, spielt es hierbei keine Rolle, welche der beiden Kurven wir reparametrisieren.
\end{Def}

\begin{Def}
	Abstände auf Graphen
	Wir definieren den \textit{gerichteten Graph-Abstand} $ \vec{\delta}_G $ als
	$$ \vec{\delta}_G(G_1,G_2) = \inf_{s: G_1 \to G_2} \: \max_{e \in E_1} \delta_F(e, s(e)) $$
	und den \textit{gerichteten schwachen Graph-Abstand} $ \vec{\delta}_{wG} $ als
	$$  \vec{\delta}_{wG}(G_1,G_2) = \inf_{s: G_1 \to G_2} \: \max_{e \in E_1} \vec{\delta}_{wF}(e, s(e)), $$
	wobei s sich über alle Graph-Zuordnungen erstreckt.
	\\
	\\The connectivity of a graph is an important measure of its resilience as a network.
\end{Def}

\begin{Def}
	$\varepsilon$-Platzierung (engl. $\varepsilon$-Placement)
	Eine \textit{$\varepsilon$-Platzierung eines Knotens $v \in V_1$} ist eine maximal zusammenhängende Komponente von $G_2$ eingeschränkt auf $B_{\varepsilon}(v)$.
	Eine \textit{$\varepsilon$-Platzierung einer Kante $e = \{u,v\} \in E_1$} ist ein Weg $P$ in $G_2$, welcher eine Platzierung $C_u$ von $u$ mit einer Platzierung $C_v$ von $v$
	so verbindet, dass $\delta_F(e, P) \leq \varepsilon$.
	\\
	In diesem Fall nennen wir $C_u$ und $C_v$ \textit{voneinander erreichbar}.
	\\
	\\
	Eine schwache $\varepsilon$-Platzierung einer Kante $e = \{u,v\} \in E_1$ ist ein Weg $P$ in $G_2$, welcher eine Platzierung $C_u$ von $u$ mit einer Platzierung $C_v$ von $v$
	so verbindet, dass $\delta_{wF}(e, P) \leq \varepsilon$.
	\\
	In diesem Fall nennen wir $C_u$ und $C_v$ \textit{schwach voneinander erreichbar}.

\end{Def}

TODO: Wir müssen irgendwo den Begriff des geometrischen Graphen einführen.
Wir müssen darüber hinaus entscheiden, ab welchem Punkt wir nur noch von geometrischen Graphen ausgehen.

TODO: Wir müssen den Begriff des Min-Sum-Graph-Abstands definieren

\newpage
Nun haben wir ausreichend Grundlagen, um uns dem min-sum graph distance widmen zu können.
\begin{Def}
	Min-Sum-Graph-Abstand (engl. min-sum graph distance)

	Allgemein bezei

\end{Def}
Im weiteren Verlauf werden wir, wenn nicht anders ausgewiesen, für den min-sum graph distance die (weak) Fréchet Distance als Abstandsmaß implizieren.

\newpage
NP-Schwerheit des Min-Sum Graph-Abstands

Bevor wir uns mit dem erwähnten Polynomialzeit-Algorithmus beschäftigen, wollen wir die NP-Schwerheit der min-sum graph distance für allgemeine Graphen zeigen.
Ein ähnliches Setting finden wir in [2] (Seite). Hier wird die NP-Schwerheit des Graph-Abstands für allgemeine Graphen über eine Reduktion von
\textit{Binary Constraint Satifaction Problem (CSP)} gezeigt. In unserem Beweis für NP-Schwerheit orientieren wir uns zu wesentlichen Teilen an der Konstruktion in [2].

\begin{Def}
	Binary Constraint Satisfaction Problem
	Ein \textit{Binary Constraint Satisfaction Problem} beschreibt folgendes Entscheidungsproblem:
	\\
	Gegeben eine Instanz $\langle X,D,C \rangle$ bestehend aus
	$$ \text{einer Menge von \textit{Variablen }}X = \{x_1, x_2, ..., x_n\},$$
	$$ \text{einer Menge von \textit{Domänen }}D = \{D_1, D_2, ..., D_n\} $$
	$$ \text{und einer Menge von \textit{Bedingungen }}C = \{C_1, C_2, ..., C_k\}. $$
	Für jede Variable $ x_i \in X$ beschreibt die Domäne $ D_i \in D$ die Menge ihrer möglichen Wertzuweisungen.
	Eine Bedingung $C_{i,j} \in C$ spezifiert für je zwei unterschiedliche Variablen $x_i, x_j \in X$ eine Relation $R_{C_{i,j}}$ $\subseteq D_v \times D_w$.
	Wir nennen ein Wertepaar $(d_i, d_j) \in D_i \times D_j$ für eine vorhandene Bedingung $C_{i,j}$ an die Variablen $x_i,x_j$ \textit{zulässig}, wenn $(d_i,d_j) \in R_{C_{i,j}}$,
	ansonsten \textit{verletzt} das Wertepaar $(d_i, d_j)$ die Bedingung $C_{i,j}$ und wir nennen $(d_i,d_j)$ \textit{unzulässig}.
	\\
	Die Fragestellung ist nun, ob alle Variablen einem Wert aus ihrer Domäne zugewiesen werden können, so dass keine Bedingung verletzt wird?
	\\
	\\
	TODO: Referenz, dass CSP NP-Schwer ist

\end{Def}

\begin{Satz}
	Theorem: Sei $\varepsilon \geq 0$. Das Entscheidungsproblem \textit{Min-Sum-Graph-Abstand}$_{(w)F}(G_1, G_2)  \leq  \varepsilon, $ ist NP-Schwer.
\end{Satz}

Sei $\langle X,D,C \rangle$ ein beliebiges Binary Constraint Satisfaction Problem.
\\
\\
Wir repräsentieren jede Variable $x_i \in X$ durch einen Knoten $v_i \in V_1$ in $G_1$ und für jede Bedingung $C_{i,j}$ an die Variablen $x_i, x_j$
hat $G_1$ die Kante $\{v_i, v_j\}$. Wir setzen $\varepsilon = |E_1|$.
\\
Sei $B_1(v_k)$ der Ball mit dem Radius $r=1$ um einen Knoten $v_k$.
Wir betten $G_1$ so in die euklidische Ebene ein, dass sich für je zwei Knoten $x_i,x_j$ ihre $\varepsilon$ Bälle nicht berühren und
der $\varepsilon$-Schlauch zwischen $x_i$ und $x_j$ keinen $\varepsilon$-Ball eines anderen Knoten überlappt.
\\
\\
Dies lässt sich z.B. realisieren, indem wir die Knoten in $G_1$ (gleichmäßig) auf einem Kreis mit entsprechend großem Radius verteilen.

Jeden Wert $d_{i,a} \in D_i$ repräsentieren wir durch einen Knoten $u_{i,a}$ in $G_2$, welchen wir innerhalb des Balles $B_1(v_i)$ einbetten.
Für jedes Wertepaar $d_{i,a} \in D_i, d_{j,b} \in D_j$ enthält $G_2$ genau dann die Kante $\{u_{i,a},u_{j,b}\}$, wenn die Wertekombination
durch eine Bedingung an die Variablen $x_i$ und $x_j$ erlaubt ist.
\\
Alle einzubettenen Kanten ergeben sich direkt (als Strecken) zwischen den jeweiligen Knoten-Einbettungen.
Insgesamt lässt sich die oben beschriebene Einbettung von $G_1$ und $G_2$ in Polynomialzeit berechnen.
\\
\\
TODO: Figure XX zeigt die Konstruktion anhand eines kleinen Beispiels
\\
\\
Sei $T_1(v_i,v_j)$  TODO: Röhre der Strecke definieren...
\\
\\
Wir wollen nun zeigen, dass
$$ \min_{s: G_1 \to G_2} \sum_{e \in E_1} \delta_{(w)F}(e, s(e)) \leq |E_1| * \tilde{\varepsilon} \iff \langle X,D,C \rangle \text{ ist lösbar}$$
"$\Leftarrow$":
Sei $\langle X,D,C \rangle$ lösbar.
\\
Dann existiert eine Wertezuweisung $(\tilde{d_1},\tilde{d_2},...,\tilde{d_n}) \in {D_1 \times D_2 \times ... \times D_n},$ welche keine der Bedingungen verletzt.
\\\\
Wir definieren eine Abbildung $\tilde{s}:G_1 \to G_2$ mit $\tilde{s}(x_i) = \tilde{d_i}$.
\\
\\
Da jeder Knoten $x_i$ in $G_1$ durch $\tilde{s}$ auf einen Knoten $G_2$ abgebildet wird, liegt offensichtlich $\tilde{s}(x_i)$ in $G_2$.
\\
Zusätzlich impliziert eine Kante $\{v_i, v_j\}$ in $G_1$ eine Bedingung $C_{i,j}$ an die Variablen $x_i$ und $x_j$ in $\langle X,D,C \rangle$
Da $(\tilde{d_1},\tilde{d_2},...,\tilde{d_n})$ keine Bedingung verletzt, ist $(d_i,d_j) \in R_{C_{i,j}}$ und damit existiert per Konstruktion
die Kante ${\tilde{d_i}, \tilde{d_j}}$ in $G_2$.
Insgesamt ist damit $\tilde{s}$ eine Graph-Zuordnung.
\\
\\
Da $\tilde{s}$ Kanten in $G_1$ auf Kanten in $G_2$ abbildet, ist für jede Kante $\{x_i, x_j\}$ in $G_1$ ihr Bild $\tilde{s}(\{x_i,x_j\})$ per Konstruktion
von $G_1$ und $G_2$ eine $\tilde{\varepsilon}$-Kanten-Platzierung, damit gilt
$$\delta_{(w)F}(\{x_i,x_j\}, \tilde{s}({x_i,x_j})) \leq \tilde{\varepsilon}.$$
\\
Daraus folgt weiter, dass $$\sum_{e \in E_1} \delta_{(w)F}(e, s(e)) \leq |E_1| * \tilde{\varepsilon}. (i) $$
\\
Über die Identität $$\min_{s: G_1 \to G_2} \sum_{e \in E_1} \delta_{(w)F}(e, s(e)) \leq \sum_{e \in E_1} \delta_{(w)F}(e,\tilde{s}(e)) $$
und (i) erhalten wir
$$\min_{s: G_1 \to G_2} \sum_{e \in E_1} \delta_{(w)F}(e, s(e)) \leq |E_1| * \tilde{\varepsilon}$$
\\
"$\Rightarrow$":
\\
Sei $$\min_{s: G_1 \to G_2} \sum_{e \in E_1} \delta_{(w)F}(e, s(e)) \leq {\varepsilon}.$$
\\
\\
So ist $s(e)$ für alle $e \in E_1$ eine (schwache) $\varepsilon$-Platzierung von $e$,
da ansonsten $\delta{(w)F}(\tilde{e}, s(\tilde{e})) > \varepsilon$ für ein $\tilde{e} \in E_1$ und damit insbesondere $\sum_{{e}\in E_1} \delta_{(w)F}(e, s(e)) > \varepsilon$.
(*)
\\
\\
Zusätzlich wollen wir argumentieren, dass $s(e)$ für alle $e \in E_1$ eine (schwache) 1-Platzierung von $e$ ist.
\\
Angenommen, für eine Kante $\{v_i,v_j\} \in E_1$ sei $s(\{v_i, v_j\})$ keine (schwache) 1-Platzierung von $\{v_i,v_j\}$.
So ist per Definition $\delta_{(w)F}(\{v_i,v_j\}, s(\{v_i, v_j\})) > 1$ und es trifft genau einer von 3 Fällen zu:
\begin{enumerate}
	\item[1)] $s(v_i) \notin B_1(v_i)$ und $s(v_j) \in B_1(v_j)$
	\item[2)] $s(v_i) \in B_1(v_i)$ und $s(v_j) \notin B_1(v_j)$
	\item[3)] $s(v_i) \notin B_1(v_i)$ und $s(v_j) \notin B_1(v_j)$
\end{enumerate}
\\
Wir betrachten zunächst Fall 1:
\\
Wegen (*) liegt $s(v_i)$ innerhalb von  einer $\varepsilon$-Platzierung von $v_i$, die wir mit $C_i$ bezeichnen.
Per Konstruktion hat $C_i$ eine Teilkomponente, welche in $B_1(v_i)$ liegt und gerade der zugehörigen $1$-Platzierung von $v_i$ entspricht und die wir mit $\tilde{C_i}$ bezeichnen.
"\\
\\
Wir defninieren $\tilde{s}: G_1 \rightarrow G_2$ wie folgt:
\begin{enumerate}
	\item[(i)] $\tilde{s}$ bildet $v_i$ auf einen beliebigen Punkt innherhalb von $\tilde{C_1}$ ab.
	\item[(ii)] Für alle anderen Knoten $v \in V_1$ ist $\tilde{s}(v)=s(v)$.
\end{enumerate}
\\
$\tilde{s}$ ist eine Graph-Abbildung, da $s$ nach Vorraussetzung eine Graph-Abbildung ist und es sich bei $\tilde{s}(v_i)$ lediglich um eine Verschiebung
des Bildpunktes von $s(v_i)$ entlang einer zusammenhängenden Komponente von $G_2$ handelt.
\\
Ferner gilt:
\begin{enumerate}
	\item[] $\delta_{(w)F}(e, \tilde{s}(e)) \leq \delta_{(w)F}(e, s(e))$, falls $v_i$ ein Knoten von $e$ ist, insbesondere ist
	\item[] $\delta_{(w)F}(e, \tilde{s}(e)) < \delta_{(w)F}(e, s(e))$, falls $e = \{v_i, v_j\}$ und in allen anderen Fällen gilt
	\item[] $\delta_{(w)F}(e, \tilde{s}(e)) = \delta_{(w)F}(e, s(e))$.
\end{enumerate}
Insgesamt folgt, dass
$$\sum_{{e}\in E_1} \delta_{(w)F}(e, \tilde{s}(e)) < \sum_{{e}\in E_1} \delta_{(w)F}(e, s(e)),$$
was im Widerspruch zu der Annahme steht, dass $s$ eine $\text{Min-Sum-Graph-Zuordnung}_{(w)F}$ ist.
\\
Die Argumentation für Fall 2 verläuft analog wie für Fall 1.
Für Fall 3 kombinieren wir die Argumentationen von Fall 1 und Fall 2, in welcher die scharfe Ungleichung im Allgemeinen allerdings erst gilt,
nachdem beide Bildpunkte $s(v_i)$ und $s(v_j)$ entsprechend versetzt wurden.
\\
\\
Damit ist für eine Kante $\{v_i, v_j\} \in E_1$ ihr Bild $s(\{(v_i, v_j)\})$ stets eine 1-Platzierung von $\{v_i, v_j\}$. (**)
\\
\\
Im Falle des Fréchet-Abstands gilt damit, dass ein $\text{Min-Sum-Graph-Mapping}_F$ $s: G_1 \to G_2$



\newpage\null\thispagestyle{empty}\newpage
%   Eventuell eine Leerseite vor der Literaturangabe, falls diese sonst auf einer Rückseite angegeben würde. (Seitennummer muss ungerade sein!)

\begin{thebibliography}{4}
	\bibitem{Buchin}
		Maike Buchin, Bernhard Kilgus. Distance Measures for Embedded Graphs - Optimal Graph Mappings.
		\textit{European Workshop on Computational Gemoetry,} 2020.

	\bibitem{Akitaya}
		Hugo Akitaya, Maike Buchin, Bernhard Kilgus, Stef Sijben, Carola Wenk. Distance measures for embedded graphs
		\textit{Computational Geometry,} Volume 95, 2021.

	\bibitem{Alt}
		Hemlut Alt, Michael Godau. Computing the Fréchet distance between two polygonal curves.
		\textit{Int. Journal of Computational Geometry and Applications,} 5:75-91, 1995.

	\bibitem{Rote}
		Günter Rote, Lexicographic Fréchet distance.
		\textit{European Workshop on Computational Gemoetry,} 2014.

\end{thebibliography}

% Symbolverzeichnis definieren
%\nomenclature{$c$}{Speed of light in a vacuum inertial frame}
%\nomenclature{$h$}{Planck constant}

%\printnomenclature
\end{document}
